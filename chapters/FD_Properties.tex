While in \autoref{chp:arrival_times} \acp{FD} have mainly been used to determine the arrival time of \acp{ICME} at \SI{1}{\AU} and Mars, other properties of \acp{FD} have been neglected in these studies. However, some empirical relations between different properties of a \ac{FD} at Earth as well as between the \ac{FD} and properties of the associated \ac{ICME} are known, as shown e.g. by \citet{Belov-2008-FD,Abunin-2012-FD}. For example, there is a clear correlation between the relative \ac{FD} amplitude or the maximum decrease rate (``steepness'') with the product of the maximum solar wind speed and maximum magnetic field. This parameter $v_\text{max} \cdot B_\text{max}$ can be used to describe the intensity of the disturbance in the solar wind. The correlation is shown in Figure 8 of \citet{Belov-2008-FD} and Figure 7 of \citet{Abunin-2012-FD}.
However, solar wind plasma and magnetic field measurements at Mars are only available from the \ac{MAVEN} spacecraft since 2014, and as explained in \autoref{sec:motivation}, it does not observe the upstream solar wind continuously. Consequently, the determination of maximum values for $B$ and $v$, which would be necessary for the validation of this relation at Mars, may be complicated for many events.

Instead, in this study, we focus on the correlation of two parameters of the \acp{FD} themselves: the relative amplitude and the maximum decrease rate. These parameters are already known to be correlated at Earth, as seen in Figure 7 of \citet{Belov-2008-FD} and Figure 5 of \citet{Abunin-2012-FD}. We use our catalog of \acp{FD} from \citet{Forstner-2019}, as well as the larger catalog by \citet{Papaioannou-2019-FD-Earth-Mars} to reproduce this relation at Mars. Consulting the analytical \ac{FD} models, \acs{PDB} and \acs{ForbMod}, which were introduced in \autoref{sec:forbush}, it becomes possible to interpret the difference between the two observed relations as a result of the expansion of the interplanetary structures.

The following article is reproduced from \textcite{Forstner-2020} from Journal of Geophysical Research: Space Physics, \copyright American Geophysical Union, under the Creative Commons CC-BY (\ccLogo\ccAttribution) license:\\

\noindent\pubcite{Forstner-2020}
\hfill Own contribution: 80\%

\newpage
\newcounter{includepdfpageJGRTwenty}

\addtocounter{section}{1}
\setcounter{subsection}{1} 
\phantomsection
\addcontentsline{toc}{section}{\arabic{chapter}.\arabic{section} Comparing the Properties of ICME-Induced Forbush Decreases at Earth and Mars (Publication JGR--Space Physics 2020)}
%
\phantomsection
\addcontentsline{toc}{subsection}{\arabic{chapter}.\arabic{section}.\arabic{subsection} Introduction}
\label{sec:paper_forstner2020}
\includepdf[pages={1}, link, linkname=paper_forstner2020, scale=.9, pagecommand={\refstepcounter{includepdfpageJGRTwenty}\label{paper_forstner2020.\theincludepdfpageJGRTwenty}}]{publications/Forstner_et_al-2020-JGRSpace.pdf}
%
\addtocounter{subsection}{1} 
\phantomsection
\addcontentsline{toc}{subsection}{\arabic{chapter}.\arabic{section}.\arabic{subsection} Data Sources and Catalogs}
\includepdf[pages={2-5}, link, linkname=paper_forstner2020, scale=.9, pagecommand={\refstepcounter{includepdfpageJGRTwenty}\label{paper_forstner2020.\theincludepdfpageJGRTwenty}}]{publications/Forstner_et_al-2020-JGRSpace.pdf}
%
\addtocounter{subsection}{1} 
\phantomsection
\addcontentsline{toc}{subsection}{\arabic{chapter}.\arabic{section}.\arabic{subsection} Definitions and Methodology}
\includepdf[pages={6-7}, link, linkname=paper_forstner2020, scale=.9, pagecommand={\refstepcounter{includepdfpageJGRTwenty}\label{paper_forstner2020.\theincludepdfpageJGRTwenty}}]{publications/Forstner_et_al-2020-JGRSpace.pdf}
%
\addtocounter{subsection}{1} 
\phantomsection
\addcontentsline{toc}{subsection}{\arabic{chapter}.\arabic{section}.\arabic{subsection} Results and Discussions}
\includepdf[pages={8-16}, link, linkname=paper_forstner2020, scale=.9, pagecommand={\refstepcounter{includepdfpageJGRTwenty}\label{paper_forstner2020.\theincludepdfpageJGRTwenty}}]{publications/Forstner_et_al-2020-JGRSpace.pdf}
%
\addtocounter{subsection}{1} 
\phantomsection
\addcontentsline{toc}{subsection}{\arabic{chapter}.\arabic{section}.\arabic{subsection} Conclusions and Outlook}
%
\addtocounter{subsection}{1} 
\phantomsection
\addcontentsline{toc}{subsection}{\arabic{chapter}.\arabic{section}.\arabic{subsection} Appendix A: Location of $m_{\text{max}}$ Within the ICME Substructures}
\includepdf[pages={17-18}, link, linkname=paper_forstner2020, scale=.9, pagecommand={\refstepcounter{includepdfpageJGRTwenty}\label{paper_forstner2020.\theincludepdfpageJGRTwenty}}]{publications/Forstner_et_al-2020-JGRSpace.pdf}
%
\addtocounter{subsection}{1} 
\phantomsection
\addcontentsline{toc}{subsection}{\arabic{chapter}.\arabic{section}.\arabic{subsection} References}
\includepdf[pages={19-21}, link, linkname=paper_forstner2020, scale=.9, pagecommand={\refstepcounter{includepdfpageJGRTwenty}\label{paper_forstner2020.\theincludepdfpageJGRTwenty}}]{publications/Forstner_et_al-2020-JGRSpace.pdf}