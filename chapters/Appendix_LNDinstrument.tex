\chapter{LND instrument }
\label{chp:GCS_Python}

\section{temperature variation of the LND instrument from 2019 to 2023}

Inside of the sensor head, LND has several temperature measusment sensor, monitoring the temperature variation of the chips and the surrounding temperature, as shown in Fig.\ref{}. The temperature sensor only operates together with LND on the day time. 
The temperature variation on the lunar surface

\begin{figure}
    \centering
    \includegraphics{images/lnd_temperature.png}
    \caption{LND temperature variations}
    \label{}
\end{figure}


\section{SEP list of LND up to 2022, December}

\begin{figure}
    \centering
    \includegraphics{images/lnd_proton_flux.png}
    \caption{LND proton flux}
    \label{}
\end{figure}

The above figures are generated by the LND webplotter which is a web application that is used to visualize the LND measurement in a quick and fast way. 
Currently, the webplotter is running in a server named etsasa located in the IEAP.
The webplotter is written in Python and the source code is available in the github repository \url{https://gitlab.physik.uni-kiel.de/LND/lnd_webplotter}. 
Inner used webplotter 
Created by Zigong Xu. following the tempelate of SOLO loader

2021 August, Lid problem. Beside, malfunction of the LND lid.


\section{script convert from }

\begin{figure}
    \centering
    \includegraphics{images/lnd_bias_current.png}
    \includegraphics{images/lnd_bias_voltage.png}
    \caption{}
    \label{}
\end{figure}


    Due to the noise of the detectors, the bias current reach the limit of the meaurements at 2$\mu$A. 

Below are the script from Stephan, which is used to calculate the correct bias voltage and current that measured on the LND instrument.

function Ibias() {
    a = $(HK_Ibias+3)
    if (a>4000) a = (147.3 + degC($(HK_T_LVPS+3))*0.164 - $(HK_Vbias+3)*0.05488) / 0.01810
    return a*0.4928
}

function Vbias() {
    return $(HK_Vbias+3)*0.05488 + $(HK_Ibias+3)*0.01810
}

\section{Discrepanchy between LND and CREME}

\section{Heavy ion spectra}

\section{He3 spectra}
Mostly the instrument effect, the He3 spectra is not reliable.

\section{L1 responce simulation}

