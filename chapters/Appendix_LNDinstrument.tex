\chapter{LND instrument }
\label{chp:GCS_Python}
\section{SEP list of LND up to 2022, December}

\missingfigure{Overall proton temperal profile} or the seperate one.

The above figures are generated by the LND webplotter which is a web application that is used to visualize the LND measurement in a quick and fast way. 
Currently, the webplotter is running in a server named etsasa located in the IEAP.
The webplotter is written in Python and the source code is available in the github repository \url{https://gitlab.physik.uni-kiel.de/LND/lnd_webplotter}. 
Inner used webplotter 
Created by Zigong Xu. following the tempelate of SOLO loader



\section{script convert from }
Below are the script from Stephan, which is used to calculate the correct bias voltage and current that measured on the LND instrument.

function Ibias() {
    a = $(HK_Ibias+3)
    if (a>4000) a = (147.3 + degC($(HK_T_LVPS+3))*0.164 - $(HK_Vbias+3)*0.05488) / 0.01810
    return a*0.4928
}

function Vbias() {
    return $(HK_Vbias+3)*0.05488 + $(HK_Ibias+3)*0.01810
}

\section{Discrepanchy between LND and CREME}

\section{Heavy ion spectra}

\section{He3 spectra}
Mostly the instrument effect, the He3 spectra is not reliable.

\section{L1 responce simulation}

