\thispagestyle{empty}

\hfill

\vfill

\noindent\myName:\\
\textit{\myTitle \\ \mySubtitle} \\%\myDegree, 
\textcopyright\ \myTime

\vspace{1cm}

\noindent\spacedlowsmallcaps{Titelbild / Cover picture}: \\[2mm]
\begin{minipage}{\textwidth}
\renewcommand{\thempfootnote}{\arabic{mpfootnote}}
{\footnotesize \foreignlanguage{ngerman}{Das Titelbild zeigt eine Skizze der Sonnenkorona während der totalen Sonnenfinsternis am 18.07.1860 von F. A. Oom. Es handelt sich dabei vermutlich um eine der ersten Beobachtungen eines koronalen Massenauswurfs (\acs{CME}) überhaupt, und zusammen mit der im Dezember 2020\footnote{see \url{https://apod.nasa.gov/apod/ap210107.html}} eine der wenigen solchen Beobachtungen während einer Sonnenfinsternis. Basierend auf \citet[S. 551]{Ranyard-1879}.}  /
The cover picture shows a drawing of the solar corona by F. A. Oom during the total solar eclipse of July 18, 1860. This is likely one of the first observations of a \ac{CME}, and also until December 2020\footnotemark[1] one of the few such observations during a solar eclipse. Adapted from \citet[page 551]{Ranyard-1879}.}
\end{minipage}\\\

\vspace{1cm}

\noindent\spacedlowsmallcaps{Erster Gutachter (Supervisor)}: \\
\myProf \\

\noindent\spacedlowsmallcaps{Zweiter Gutachter %(Advisor)
}: \\
\myOtherProf


\vspace{1cm}

\noindent\spacedlowsmallcaps{Tag der m\"undlichen Pr\"ufung}: \\
\myExamDate

\bigskip

\noindent\spacedlowsmallcaps{Zum Druck genehmigt}: \\

\vspace{1cm}

%\noindent gez. ..., Dekan
