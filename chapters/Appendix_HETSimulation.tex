\chapter{Isotropic simulation of the High Energy Telescope with the Solar Orbiter spacecraft model}
\label{chp:HETSimulation}

To simulate the response of the High Energy Telescope (HET, \autoref{sec:solohet}), part of the Energetic Particle Detector \citep[EPD][]{RodriguezPacheco-2019-EPD}, for an isotropic radiation field, a simulation using the GEometry And Tracking 4 \citep[Geant4][]{Agostinelli-2003} was performed. GEANT4 is a software toolkit developed at CERN for Monte Carlo simulations of the interaction of particles with matter, and is widely used in many different fields. A Geant4 simulation typically requires the definition of a particle source (e.g. a particle beam, a surface or volume source), a model of the geometry and materials of the experimental setup, and one or more sensitive detectors in which particle hits are detected. A so-called ``physics list'' describes all possible interaction processes and their probabilities, from which Geant4 then chooses stochastically for each simulated particle. As a result, the trajectory of each particle and the detected particles in each detector can be stored and used processed to calculate e.g. the response function of a particle detector.

This simulation builds on top of the work by \citet{Elftmann-2020-PhD}, who simulated the nominal data products of HET with Geant4. In this case, particles were simulated only from a circular source in front of the A detector \citet[Figure 5.1]{Elftmann-2020-PhD}, which fills the nominal field of view (FOV) of HET. However, particles entering HET from outside its FOV may also play a role for certain data products, especially for single-detector counters, which are sensitive to particles entering the telescope from all directions (e.g., through the housing). These counters were used by \citet{Forstner-2021-SolO} to observe Forbush decreases with HET. Thus, it makes sense to simulate the HET detector in an isotropic particle flux to model its response to Galactic Cosmic Rays (GCRs), and this will be done in \autoref{sec:isotropic_sim}.

For an isotropic particle flux, it is also important to take into account the Solar Orbiter spacecraft that HET is mounted on. This is less relevant for the nominal FOV, as the HET telescopes are oriented so that their openings do not point towards the spacecraft body (i.e. tangential to an edge of the body). But for single detector counters, particles coming from the solid angle covered by the spacecraft can be shielded away, or may generate secondaries that are then detected within HET. This will be investigated in \autoref{sec:spacecraft_model}.

\section{Isotropic simulation of HET}
\label{sec:isotropic_sim}

For the isotropic simulation, the geometric model of the HET instrument and its electronics box were reused from the simulation by \citet{Elftmann-2020-PhD}, so the reader is referred to this work for further details about its definition. The simulation setup was also very similar to \citet{Elftmann-2020-PhD}, using Geant4 version 10.1.2 with the pre-defined general-purpose physics list \texttt{QGSP\_BERT}, and a power law spectrum for the energy-dependent intensity $I$ of the input particles:
\begin{equation}
I(E) \sim E^{-1}
\end{equation}
This makes it possible to simulate the same number of particles in each primary energy bin when the bins are logarithmically spaced. Only protons between \SI{5}{\mega\electronvolt} and \SI{100}{\giga\electronvolt} were simulated as input particles in this case, as other species were neglected in the modeling approach of \citet{Forstner-2021-SolO}. Still, all secondary particles generated by these primary protons are taken into account in the output of simulation. Of course, other primary species such as electrons and heavy ions can be simulated in a similar fashion in the future.

In contrast to the circular planar source surface used by \citet{Elftmann-2020-PhD}, a spherical source with a radius of \SI{15}{\centi\meter} (large enough to surround HET and its electronics box) was defined, and particles were injected following a cosine-law angular distribution. This represents an isotropic flux entering HET from all sides.
The simulation was run for \num{5e8} particles, and $\sim\num{1.6e7}$ particle hits (primary or secondary) were registered in the HET detectors.

\section{The spacecraft model}
\label{sec:spacecraft_model}

To simulate how the interaction of GCR particles with the Solar Orbiter spacecraft affects the HET measurements, the spacecraft needs to be included into the Geant4 simulation. Accurately modeling a spacecraft is notoriously difficult, as it consists of numerous components and information about their exact shape and composition is not always readily available. Thus, simulations taking into account spacecraft effects often need to make many assumptions and drastically simplify the geometry of the spacecraft (see e.g. \citet{Appel-2018-PhD,Appel-2018} for a similar simulation of the Mars Science Laboratory rover).

In the case of Solar Orbiter, ESA has provided a CAD model of the spacecraft that can serve as a reference for simulations (D. Müller, 2020, priv. comm.). This file (\texttt{dpl7a1\_bulk.bdf}) is in the NASTRAN BDF format.

\url{http://www.f-e-x.com/}

\begin{figure}
	\centering
	\includegraphics[width=0.7\linewidth]{images/solo_spacecraft_model}
	\caption[Density model of the Solar Orbiter spacecraft]{Density model of the Solar Orbiter spacecraft as supplied by ESA. Colored tetrahedra correspond to mass points that are inserted into the model as a replacement for structures whose geometry is not modeled (e.g. scientific payload).}
	\label{fig:solo_spacecraft_model}
\end{figure}

To speed up future simulations, the spacecraft model may be converted into spherical shells of equivalent column density that are placed around HET.

\section{Analysis of the simulation results}
