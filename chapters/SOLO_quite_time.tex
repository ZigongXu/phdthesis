With the launch of the \ac{SolO} on Februbar 10, 2020, we have further expanded our measurement ability of energetic particles in the new era of space observations and the inner heliosphere. This is achieved by the \ac{EPD} onboard \ac{SolO}, measuring particles in the energy range over several orders of magnitude from keV to GeV.

In the first year of its mission, \ac{SolO} has observed numerous events and structures associated with solar eruptions. \citet{GomezHerrero-2021-SolO} reported the first near-relativistic solar electron events observed by \ac{EPD} on July 2020. The solar origin and the interplanetary transport condition of those particles are investigated. Later, \citet{Kolhoff2021AA} analyzed the first widespread \ac{SEP} of \ac{SC} 25 that occurred on November 29, 2020. \citet{Mason2021AA} analyzed the heavy ion properties of this event. Besides, \citet{Mason2021AA_he3rich} reported the observations of five impulsive \acp{SEP} during the first perihelion movement of \ac{SolO}. Those events have low intensity and are undetectable at one au, indicating the special advantage of \ac{SolO} and \acs{PSP} in the inner heliosphere. Furthermore, \ac{SolO} also observed the energetic particles associated with the \acp{CIR} \citep{Allen2021AA_suprathermal}, \acp{SIR} \citep{Aran2021AA} and an \acp{ICME} \citep{Kilpua2021AA}.


Additionally, the second half of the solar activity minimum provides a valuable quiet period in 2020. This period is characterized by the absent or rare occurrence of \acp{SEP}, interplanetary shocks, and other related activities, apart from the abovementioned events. Therefore, studying the ion spectra and their variation in the inner heliosphere during this period is an excellent opportunity to understand \acp{ACR}, \acp{GCR} and properties of quiet time spectra. 
This work attempts to answer the first two questions we asked in Sec.~\ref{sec:Motivation}: "How do the \acp{ACR} and \ac{GCR} behave during the most recent solar activity minimum and the onset phase of solar cycle 25?" and "What is the difference between the current solar cycle and the previous one?"

\subsection*{The overview of the publication}

Below we list the main observations from \citet{Mason-2021-SolOQuietTime}.

\begin{itemize}
    \item The quiet-time ion spectra observed by the \ac{SIS} and the \ac{HET} during the solar minimum between Feb 2020 and Jan 2021 are shown. The spectra are comprised of protons, helium-4, helium-3, oxygen, carbon, and iron and include \acp{ACR}, \acp{GCR}, and lower energy particles from impulsive \ac{SEP}. The energy range spans a few tens of kev/nuc - $\sim$ GeV/nuc,
    \item Super-quiet periods are defined, and the corresponding spectra are derived. The overall shape and intensity are similar to the quiet time spectra except that the lower energy helium-4 spectrum extends to $\sim$ 300 keV/nuc before the intensity rises.
    \item The radial dependence of 4.4 Mev/nuc He and O during this period was first derived. The gradient has considerable uncertainty but is consistent with a positive small O gradient.
\end{itemize}
The following article is reproduced from \textcite{Mason-2021-SolOQuietTime} with permission from Astronomy \& Astrophysics, \copyright  ESO:\\


\noindent\pubcite{Mason-2021-SolOQuietTime}\\
\strut\hfill Own contribution: 25\%

\newpage
\newcounter{includepdfpageAATwentyOne}

\addtocounter{section}{1}
\setcounter{section}{1} 
\phantomsection
% \addcontentsline{toc}{section}{\arabic{chapter}.\arabic{section} Primary and albedo protons detected by the Lunar Lander Neutron and Dosimetry experiment on the lunar farside(Publication Frontier in Astronomy and Space Sciences 2022)}
% %
\phantomsection
\addcontentsline{toc}{section}{\arabic{chapter}.\arabic{section} Introduction}
\label{sec:paper_mason2021}
\includepdf[pages={1}, link, linkname=paper_mason2021, scale=.9, pagecommand={\refstepcounter{includepdfpageAATwentyOne}\label{paper_mason2021.\theincludepdfpageAATwentyOne}}]{publications/Mason_et_al_2021_AandA.pdf}
%
\addtocounter{section}{1} 
\phantomsection
\addcontentsline{toc}{section}{\arabic{chapter}.\arabic{section} Observations}
\includepdf[pages={2}, link, linkname=paper_mason2021, scale=.9, pagecommand={\refstepcounter{includepdfpageAATwentyOne}\label{paper_mason2021.\theincludepdfpageAATwentyOne}}]{publications/Mason_et_al_2021_AandA.pdf}
%
\addtocounter{section}{1} 
\phantomsection
\addcontentsline{toc}{section}{\arabic{chapter}.\arabic{section} Discussion and conclusion}
\includepdf[pages={3}, link, linkname=paper_mason2021, scale=.9, pagecommand={\refstepcounter{includepdfpageAATwentyOne}\label{paper_mason2021.\theincludepdfpageAATwentyOne}}]{publications/Mason_et_al_2021_AandA.pdf}
%
\addtocounter{section}{1} 
\phantomsection
\addcontentsline{toc}{section}{\arabic{chapter}.\arabic{section} References}
\includepdf[pages={4}, link, linkname=paper_mason2021, scale=.9, pagecommand={\refstepcounter{includepdfpageAATwentyOne}\label{paper_mason2021.\theincludepdfpageAATwentyOne}}]{publications/Mason_et_al_2021_AandA.pdf}
%
\addtocounter{section}{1} 
\phantomsection
\addcontentsline{toc}{section}{\arabic{chapter}.\arabic{section} Appendix}
\includepdf[pages={5-6}, link, linkname=paper_mason2021, scale=.9, pagecommand={\refstepcounter{includepdfpageAATwentyOne}\label{paper_mason2021.\theincludepdfpageAATwentyOne}}]{publications/Mason_et_al_2021_AandA.pdf}
%

