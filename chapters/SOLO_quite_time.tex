
With the launch of \ac{SolO} since 2020, we further expand our measurment ability of the energetic particles in the new era of space and in the inner heliosphere. Such a break through is successfully accomplished by \ac{EPD} onboard the \ac{SolO}. A suite of telescopes have been measuring the charged particle span few orders of magnetitude from keV to GeV.

In the initial first year's mission, \ac{SolO} have observed numerous events and structures associated with solar eruptions. For instance, \citet{GomezHerrero-2021-SolO} reported the first near-relativistic solar electron events observed by \ac{EPD} on july 2020. The solar origin and the interplanetary transport condition of those particles are investigated. Later \citet{Kolhoff2021AA} carried out the research on the first widespread \ac{SEP} happened on November 29, 2020. \citet{Mason2021AA} analyzed the heavy ion properties of such event. Besides, \citet{Mason2021AA_he3rich} reported the observations of five impulsive \acp{SEP} during the first perihlion movement of \ac{SolO}. Those events have low intensity, and no detectable at 1 au, indicating the special advantage of \ac{SolO} in the inner heliosphere. Furthermore, \ac{SolO} also observed the energetic particles orients from the \ac{CIR} \citep{Allen2021AA_suprathermal}, \ac{SIR} \cite{Aran2021AA} and \ac{ICME} \citep{Kilpua2021AA}


On the other hand, the secend half of solar minimum provides a valueable quite time period. Such a period is charaterized with absent or rare occuring of \ac{SEP}, interplantary shocks and other related activities, apart from the events we aforementioned. The observation of long term varied \ac{ACR}, \ac{GCR} and short timescaled "turn-up" spectra below few MeV/nuc during this time period are free from the interference of the additional particles from sun. Therefore, it is a good opportunity to study the ion spectra and their variation in the inner heliosphere during the solar minimum.
\\
\textit{The overview of the publication}\\
Below we listed main observations from the \citep{Mason-2021-SolOQuietTime}.

\begin{itemize}
    \item The quite time ion spectra observed by SIS and HET during the solar minimum between Feb 2020 and Jan 2021 are reported. The spectra composed of protons, helium-4, helium-3, oxygen, carbon and iron. The energy range spans over few tens of Kev/nuc - $\sim$ GeV/nuc.
    \item The super-quite time periods are defined and the correponding spectra are deried. The overall shape and intensity are similar with the quite time spectra except that the lower energy helium-4 spectrum extends to $\sim$ 300 keV/muc before the intensity steeply rising.
    \item The radial dependence of 4.4 Mev/nuc He and O during this period was first derived. The gradient have larger uncertainty but is consistent with a positive small O gradient.
\end{itemize}
The following article is reproduced from \textcite{Mason-2021-SOLO} with permission from Astronomy \& Astrophysics, \copyright  ESO:\\


\noindent\pubcite{Mason-2021-SOLO}\\
\strut\hfill Own contribution: 25\%

\newpage
\newcounter{includepdfpageAATwentyOne}

\addtocounter{subsection}{1}
\setcounter{subsubsection}{1} 
\phantomsection
\addcontentsline{toc}{subsection}{\arabic{chapter}.\arabic{section}.\arabic{subsection} Primary and albedo protons detected by the Lunar Lander Neutron and Dosimetry experiment on the lunar farside(Publication Frontier in Astronomy and Space Sciences 2022)}
%
\phantomsection
\addcontentsline{toc}{subsubsection}{\arabic{chapter}.\arabic{section}.\arabic{subsection}.\arabic{subsubsection} Introduction}
\label{sec:paper_mason2021}
\includepdf[pages={1}, link, linkname=paper_mason2021, scale=.9, pagecommand={\refstepcounter{includepdfpageAATwentyOne}\label{paper_mason2021.\theincludepdfpageAATwentyOne}}]{publications/Mason_et_al_2021_AandA.pdf}
%
\addtocounter{subsubsection}{1} 
\phantomsection
\addcontentsline{toc}{subsubsection}{\arabic{chapter}.\arabic{section}.\arabic{subsection}.\arabic{subsubsection} Observations}
\includepdf[pages={2}, link, linkname=paper_mason2021, scale=.9, pagecommand={\refstepcounter{includepdfpageAATwentyOne}\label{paper_mason2021.\theincludepdfpageAATwentyOne}}]{publications/Mason_et_al_2021_AandA.pdf}
%
\addtocounter{subsubsection}{1} 
\phantomsection
\addcontentsline{toc}{subsubsection}{\arabic{chapter}.\arabic{section}.\arabic{subsection}.\arabic{subsubsection} Discussion and conclusion}
\includepdf[pages={3}, link, linkname=paper_mason2021, scale=.9, pagecommand={\refstepcounter{includepdfpageAATwentyOne}\label{paper_mason2021.\theincludepdfpageAATwentyOne}}]{publications/Mason_et_al_2021_AandA.pdf}
%
\addtocounter{subsubsection}{1} 
\phantomsection
\addcontentsline{toc}{subsubsection}{\arabic{chapter}.\arabic{section}.\arabic{subsection}.\arabic{subsubsection} References}
\includepdf[pages={4}, link, linkname=paper_mason2021, scale=.9, pagecommand={\refstepcounter{includepdfpageAATwentyOne}\label{paper_mason2021.\theincludepdfpageAATwentyOne}}]{publications/Mason_et_al_2021_AandA.pdf}
%
\addtocounter{subsubsection}{1} 
\phantomsection
\addcontentsline{toc}{subsubsection}{\arabic{chapter}.\arabic{section}.\arabic{subsection}.\arabic{subsubsection} Appendix}
\includepdf[pages={5-6}, link, linkname=paper_mason2021, scale=.9, pagecommand={\refstepcounter{includepdfpageAATwentyOne}\label{paper_mason2021.\theincludepdfpageAATwentyOne}}]{publications/Mason_et_al_2021_AandA.pdf}
%