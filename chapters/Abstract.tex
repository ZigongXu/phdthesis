% Abstract and Zusammenfassung
\pdfbookmark[1]{Abstract}{abstract}
\chapter*{Abstract}

\Acp{CME} are clouds of plasma and magnetic field regularly ejected from the Sun at high speeds that propagate out into interplanetary space. These events are one of the important space weather phenomena.
Their strong and turbulent magnetic field can cause disruptions of spacecraft electronics as well as terrestrial infrastructure, and they can be associated with the acceleration of energetic particles, which may cause increased radiation exposure, e.g. for astronauts.
To be able to validate and consequently improve theoretical models predicting their arrival at Earth and other locations in the heliosphere, it is important to employ many different data sources measuring the various signatures of these interplanetary \acp{CME} (\acused{ICME}\acsp{ICME}) at different locations, so that their temporal and radial evolution can be studied. These investigations are also significantly aided by observations from remote sensing telescopes, which can directly observe the global structure of the \acp{ICME} and track them out to large radial distances.

The studies presented in this thesis introduce Mars into the framework of routinely available locations for the in situ observation of space weather. Here, \acp{ICME} can be detected using \acl{FD} measurements by the \ac{RAD} onboard the \acl{MSL} rover \textit{Curiosity}. \aclp{FD} are short-term decreases in the \acl{GCR} flux caused by the magnetic structure of the \ac{ICME} partly shielding away the cosmic rays.
These measurements of \aclp{FD} are utilized in this thesis to detect \ac{ICME} arrival times for statistical studies of events seen at two planets, Earth and Mars, or at one of the two \acs{STEREO} spacecraft and Mars, during close longitudinal alignment.
The measurements show for the first time that fast \acp{ICME} can continue to decelerate beyond the orbit of Earth due to their interaction with the slower ambient solar wind.
Using remote sensing observations from the \acs{STEREO} heliospheric imagers, we study further \acp{ICME} that hit Mars and benchmark the accuracy of different approaches for the analysis of these heliospheric imager data.
Subsequently, the \acl{FD} data for the thereby cataloged events are further investigated to infer not only the arrival time, but also more information about the radial evolution of the \ac{ICME} properties by comparison with analytical modeling approaches.

Finally, two case studies are performed: First, the major space weather events of September 2017 and their impact on Mars are examined, including the investigation of the solar energetic particle event and three associated \acp{CME} that interacted and merged on their way towards Mars. Second, the first in situ observations of an \ac{ICME} at the Solar Orbiter spacecraft, which launched in February 2020, are presented. In this study, we describe the capabilities of the Solar Orbiter's High Energy Telescope for high-resolution observations of \aclp{FD} and use its measurements in combination with a reverse modeling approach to show that the expansion of the \ac{ICME} was non-uniform, possibly due to interaction with a following solar wind stream interaction region.


\cleardoublepage
\pdfbookmark[1]{Zusammenfassung}{zusammenfassung}
\chapter*{Zusammenfassung}

\begin{otherlanguage}{ngerman}
Koronale Massenauswürfe (\textit{\aclp{CME}\acused{CME}}, \acp{CME}) sind magnetisierte Plasmawolken, die die Sonne regelmäßig mit hoher Geschwindigkeit ausstößt und die sich anschließend im interplanetaren Raum ausbreiten. Sie gehören zu den wichtigsten Phänomenen des sogenannten Weltraumwetters.
Ihr starkes und turbulentes Magnetfeld kann für Störungen bei Satellitenelektronik sorgen oder sogar Infrastruktur auf der Erde beschädigen. Zusätzlich stehen \acp{CME} häufig auch im Zusammenhang mit der Beschleunigung von hochenergetischen Teilchen, die beispielsweise bei Astronauten für eine erhöhte Strahlendosis sorgen können.
Um theoretische Modelle, die die Ankunftszeit von interplanetaren \acp{CME} (\acused{ICME}\acsp{ICME}) an der Erde oder anderen Orten im Sonnensystem vorhersagen, besser überprüfen und daraufhin auch verbessern zu können, ist es wichtig, Daten von möglichst vielen Messinstrumenten einzubeziehen. So können unterschiedliche Merkmale von \acp{ICME} an mehreren Orten im Sonnensystem gemessen und damit deren zeitliche und radiale Entwicklung untersucht werden. Ebenso hilfreich dür diese Untersuchungen sind bildgebende Teleskope, die die globale Struktur der \acp{ICME} direkt beobachten und sie weit hinaus in den interplanetaren Raum verfolgen können.

Die in dieser Dissertation vorgestellten Forschungsarbeiten führen den Mars als weiteren durchgehend verfügbaren Beobachtungspunkt im Rahmen der In-situ-Beobachtung des Weltraumwetters ein. Hier können \acp{ICME} mithilfe von sogenannten \aclp{FD} detektiert werden, die in den Messungen des \ac{RAD} an Bord des Rovers \textit{Curiosity} (\acl{MSL}) erscheinen. Hierbei handelt es sich um kurzzeitige Abschwächungen der galaktischen kosmischen Strahlung, die die magnetische Struktur der \acp{ICME} durch Abschirmung hervorruft.
Die Messungen von \aclp{FD} werden hier verwendet, um die Ankunftszeiten von \acp{ICME}, die nacheinander Erde und Mars treffen, oder alternativ eine der \acs{STEREO}-Sonden und dann den Mars, statistisch zu untersuchen. % during close longitudinal alignment
Die Messungen zeigen zum ersten Mal, dass schnelle \acp{ICME} auch außerhalb der Erdbahn durch die Wechselwirkung mit dem umgebenden langsameren Sonnenwind weiter abgebremst werden.
Mithilfe der bildgebenden Teleskope auf \acs{STEREO}, den sogenannten Heliospheric Imagers, untersuchen wir weitere \acp{ICME} die den Mars getroffen haben und überprüfen damit die Genauigkeit verschiedener Methoden für die Analyse dieser Bilddaten.
Anschließend werden die \acl{FD}-Messungen für die so katalogisierten \acp{ICME} noch genauer untersucht, um neben der Ankunftszeit durch den Vergleich mit analytischen Modellen noch weitere Informationen über die radiale Entwicklung der \acp{ICME} zu gewinnen.

Daraufhin werden noch zwei Fallstudien vorgestellt: Zunächst werden die starken Weltraumwetter-Ereignisse im September 2017 und ihre Auswirkungen auf dem Mars vorgestellt -- hier werden neben dem solaren Teilchenereignis auch drei dazugehörige \acp{CME} untersucht, die zusammentreffen und sich auf dem Weg zum Mars vereinigen. Zuletzt werden die ersten Messungen eines \ac{ICME} mit der Raumsonde Solar Orbiter vorgestellt, die im Februar 2020 gestartet ist. In dieser Studie zeigen wir, wie mit dem High Energy Telescope an Bord von Solar Orbiter Forbush decreases mit hoher Auflösung gemessen werden können, und rekonstruieren die Daten des beobachteten Forbush decrease mit einem theoretischen Modell. Die Ergebnisse zeigen, dass der \ac{ICME} ein ungleichmäßiges Expansionsverhalten zeigt, möglicherweise durch den Einfluss einer nachfolgenden Sonnenwind-Wechselwirkungsregion (\textit{stream interaction region}).
\end{otherlanguage}