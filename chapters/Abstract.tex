% Abstract and Zusammenfassung
% 2023.08。27 about a month later after the defence, I start to correct the thesis. Bob said, it is about 1 day's work load. That means for a postdoc, you should finish this in like 8 hours. Let's count how much you spend here.
\pdfbookmark[1]{Abstract}{abstract}
\chapter*{Abstract}
Energetic particles including \acp{SEP}, \acp{GCR} and \acp{ACR}, play a crucial role in shaping the local space environment. Understanding how these energetic particles spread throughout the heliosphere, considering different particle sources and periodically varying solar modulations and their impact on radiation environments, are still subjects of ongoing debate in the scientific community. To address these questions and advance our understanding, we need new instruments that are essential to provide new measurements of energetic particles from unique perspectives.

The \ac{LND} aboard the Chang'E - 4 lander and the \ac{HET} onboard the \ac{SolO} are two new state-of-the-art high-energy particle telescopes that provide invaluable insights into the nature of local space environment. \ac{LND} is the first such instrument operating on the far-side surface of the Moon. \ac{SolO} enables precious observations of energetic particles when the satellite orbits the Sun between 0.3 - 1 au.

In the first part of this thesis, we present new measurements obtained from \ac{LND} on the lunar surface, including observations of \acp{SEP}, \acp{GCR}, and albedo protons which are a subclass of secondary particles generated by the interaction between \acp{GCR} and the lunar regolith. Notably, we report the first \ac{SEP} event ever observed on the lunar far-side surface. Using both in-situ measurements and remote sensing observations, we investigate the solar origin of these particles and discuss the potential transport mechanisms of protons and electrons for distributing these particles over a wide range of longitudes in the heliosphere. We also determine the flux of \acp{GCR} and albedo protons that were observed by \ac{LND} during the recent solar activity minimum between 2019 - 2020. Through comparison with observations from other instruments
%\ac{EPHIN} onboard \ac{SOHO} and \ac{CRaTER} onboard \ac{LRO}, 
%\ac{REDMoon},
we find the \ac{GCR} proton spectra and the albedo proton flux from \ac{LND} are consistent with the corresponding measurements from other instruments. In particular, below 50 MeV, the albedo proton flux exceeds the (primary) \ac{GCR} proton flux.
Throughout this section, attention is also devoted to the calibration of data, the validation of the instrument performance, and the generation of reliable scientific data products.

The second part of this thesis focuses on the new measurements from \ac{HET}. Firstly, the quiet time spectra of ions between 2020.02 - 2021.01 are reported. These spectra are averaged between 0.5 - 1 au and clearly show \acp{GCR}, \acp{ACR}, and the very low energy part of the spectrum. Lastly, we present preliminary results regarding the observation of \acp{ACR} and radial gradient of \ac{ACR} helium between 2020.02 and 2022.10, which is approximately 20\%/au within the energy range of 10 - 50 MeV/nuc. Overall, the findings are consistent with measurements from the \ac{PSP} within uncertainties of \ac{HET} and \ac{PSP}. Besides, the radial gradient's energy dependency and time variation are also discussed. It is worth noting that, at the moment, the results of the \ac{ACR} radial gradients are still preliminary, and some peculiar features of \acp{ACR} in this solar cycle are not well understood.



% \Acp{CME} are clouds of plasma and magnetic field regularly ejected from the Sun at high speeds that propagate out into interplanetary space. These events are one of the most important space weather phenomena.
% Their strong and turbulent magnetic field can cause disruptions of spacecraft electronics as well as terrestrial infrastructure, and they can be associated with the acceleration of energetic particles, which may cause increased radiation exposure, e.g. for astronauts.
% To be able to validate and consequently improve theoretical models predicting the arrival of interplanetary \acp{CME} (\acused{ICME}\acsp{ICME}) at Earth and other locations in the heliosphere, it is important to employ many different data sources measuring the various signatures of \acp{ICME} at different locations, so that their temporal and radial evolution can be studied. These investigations are also significantly aided by observations from remote sensing telescopes, which can directly observe the global structure of the \acp{ICME} and track them out to large radial distances.

% The studies presented in this thesis introduce Mars into the framework of routinely available locations for the in situ observation of space weather. Here, \acp{ICME} can be detected using \acl{FD} measurements by the \ac{RAD} onboard the \acl{MSL} rover \textit{Curiosity}. \aclp{FD} are short-term decreases in the \acl{GCR} flux caused by the magnetic structure of the \ac{ICME} partly shielding away the cosmic rays.
% The measurements of these \aclp{FD} are utilized in this thesis to determine \ac{ICME} arrival times for statistical studies of events seen at two planets, Earth and Mars, or at one of the two \acs{STEREO} spacecraft and Mars, during close longitudinal alignment.
% The measurements show for the first time that fast \acp{ICME} can continue to decelerate beyond the orbit of Earth due to their interaction with the slower ambient solar wind.
% Using remote sensing observations from the \acs{STEREO} heliospheric imagers, we study additional \acp{ICME} that hit Mars and benchmark the accuracy of different approaches for the analysis of these heliospheric imager data.
% Subsequently, the \acl{FD} data for the thereby cataloged events are further investigated to infer not only the arrival time, but also more information about the radial evolution of the \ac{ICME} properties by comparison with analytical modeling approaches.

% Finally, two case studies are performed: First, the major space weather events of September 2017 and their impact on Mars are examined, including the investigation of the solar energetic particle events and three associated \acp{CME} that interacted and merged on their way towards Mars. Second, the first in situ observations of an \ac{ICME} at the Solar Orbiter spacecraft, which launched in February 2020, are presented. In this study, we describe the capabilities of the Solar Orbiter's High Energy Telescope for high-resolution observations of \aclp{FD} and use its measurements in combination with a reverse modeling approach to show that the expansion of the \ac{ICME} was non-uniform, possibly due to interaction with a following solar wind stream interaction region.


\cleardoublepage
\pdfbookmark[1]{Zusammenfassung}{zusammenfassung}
\chapter*{Zusammenfassung}

\begin{otherlanguage}{ngerman}
    Energiereiche Teilchen, darunter solare energiereiche Teilchen (SEPs\acused{SEP}), galaktische kosmische Starhlung (GCRs\acused{GCR}) und anomale kosmische Strahlung (ACRs\acused{ACR}), spielen eine entscheidende Rolle für die lokale Weltraumumgebung. Die Frage, wie sich diese energiereichen Teilchen in der Heliosphäre ausbreiten, wobei verschiedene Teilchenquellen und periodisch variierende solare Modulationen sowie deren Auswirkungen auf die Strahlungsumgebung zu berücksichtigen sind, wird in der wissenschaftlichen Gemeinschaft immer noch diskutiert. Um diese Fragen anzugehen und unser Verständnis zu verbessern, benötigen wir moderne Instrumente, die neue Messungen von energetischen Teilchen aus einzigartigen Perspektiven ermöglichen.

    Das \ac{LND} an Bord des Chang'E - 4 Landers und das \ac{HET} an Bord des \ac{SolO} sind zwei solche hochmodernen Hochenergieteilchenteleskope, die unschätzbare Einblicke in die Natur der lokalen Weltraumumgebung liefern. \ac{LND} ist das erste Instrument dieser Art, das auf der Rückseite des Mondes betrieben wird. \ac{SolO} ermöglicht wertvolle Beobachtungen von energetischen Teilchen, w\"{a}hrend der Satellit die Sonne zwischen 0,3 und 1 AU umkreist.
    
    Im ersten Teil dieser Arbeit stellen wir die neuen Messungen von \ac{LND} auf der Mondoberfläche vor, einschließlich Beobachtungen von \acp{SEP}, \acp{GCR} und Albedo-Protonen, einer Unterklasse von Sekundärteilchen, die durch die Wechselwirkung zwischen \acp{GCR} und dem Mondregolith entstehen. Insbesondere berichten wir über das erste \ac{SEP}-Ereignis, das jemals auf der mondfernen Oberfläche beobachtet wurde. Anhand von In-situ-Messungen und Fernerkundungsbeobachtungen untersuchen wir den solaren Ursprung dieser Teilchen und erörtern die möglichen Transportmechanismen von Protonen und Elektronen, die die bei 1 AU beobachteten breite longitidunale Verteilung ermöglichen könnten. Wir bestimmen auch den Fluss von \acp{GCR} und Albedo-Protonen, die von \ac{LND} während des jüngsten solaren Aktivitätsminimums zwischen 2019 und 2020 beobachtet wurden. Durch den Vergleich mit Beobachtungen von anderen Instrumenten sowie mit Vorhersagen von Modellen stellen wir fest, dass die \ac{GCR}-Protonenspektren und der Albedo-Protonenfluss konsistent sind. Insbesondere unterhalb von 50 MeV übersteigt der Albedo-Protonenfluss den (primären) \ac{GCR}-Protonenfluss.
    In diesem Abschnitt wird auch auf die Kalibrierung der Daten, die Validierung der Messeigenschaften des Instruments und die Erzeugung zuverlässiger wissenschaftlicher Datenprodukte eingegangen.
    
    Der zweite Teil dieser Arbeit konzentriert sich auf die neuen Messungen von \ac{HET}. Zunächst wird über die Spektren während ruhigen Zeiten von Ionen zwischen 2020,2 und 2021,1 berichtet. Diese Spektren sind zwischen 0,5 - 1 AU gemittelt und zeigen deutlich die Existenz von \acp{GCR}, \acp{ACR} sowie den untersten Energiebereich des Spektrums. Schließlich präsentieren wir vorläufige Ergebnisse bezüglich der Beobachtung von \acp{ACR} und des radialen Gradienten von \ac{ACR} Helium zwischen 2020.2 und 2022.10, der etwa 20\%/AU im Energiebereich von 10 - 50 MeV/nuc beträgt. Insgesamt stimmen die Ergebnisse mit den Messungen von \ac{PSP} innerhalb der Unsicherheiten überein. Außerdem werden die Energieabhängigkeit des radialen Gradienten und seine zeitliche Variation diskutiert. Es ist erwähnenswert, dass die Ergebnisse zu den radialen Gradienten im Moment noch vorläufig sind und einige besondere Eigenschaften der ACRs in diesem Sonnenzyklus noch nicht gut verstanden sind.
    

\end{otherlanguage}