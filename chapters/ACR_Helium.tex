
In the end of section ~\ref{chp:SOLO_Quite_time}, \citet{Mason-2021-SolOQuietTime} reports the radial gradient of \ac{ACR} oxygen and helium-4 with energy of 4.4 MeV/nuc inside 1 au. The results are based on the measurement in 2020 and the intensities of particle are plotted as a function of three radial gradients. Though the relative larger uncertainty which is due to the limited count number, the results suggest a consistent oxygen gradient
with the observation from Helios and \ac{PSP} \citep{Rankin2021ApJ,Marquardt2018AA}.

As of 2023, \ac{SolO} has finished its fifth orbit and continue its trip in the inner heliosphere, providing more valuable measurements by \ac{EPD} for the study of \ac{ACR} radial gradient of \ac{ACR}. 
In this chapter, we will employ the new data from \ac{HET} onboard \ac{SolO} and present the first observation of the \ac{ACR} heliums in the inner heliosphere between 2020 and 2022. Possible interuptions including \acp{SEP}, periodically appeared compression regions and the long term solar modulation are properly considered. Moreover, the helium flux measured by \ac{EPHIN} onboard \ac{SOHO} is utilized to derive the ratio of \ac{SolO} and \ac{SOHO}/\ac{EPHIN} in order to remove the effect of solar modulation. Our preliminary results show the consistene helium flux level between \ac{SolO} and \ac{SOHO}/\ac{EPHIN} and the radial gradient of helium-4 is consistent with the previous results from \ac{PSP} within the uncertainty.

Below we report the details of the data analysis and summary of the results, although some results are still in the preliminary phase. We will continue the data analysis in the future and the final version of the paper is in preparation for publication.



\section{Introduction and theory background}

\section{Instrument}

\section(Observation and data analysis)

\section{Results and summary}












%\input{chapters/pub04_xu2022}
