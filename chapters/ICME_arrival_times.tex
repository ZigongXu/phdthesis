An interesting constellation to study the propagation of \acp{ICME} is the so-called opposition constellation, where two planets (or spacecraft) are closely aligned in heliospheric longitude. Near these oppositions, CMEs that are seen in situ at one location are most likely to be seen by the other as well, which allows to investigate their radial evolution.

Previous studies, such as the model proposed by \citet{Gopalswamy-2001} --- a predecessor of the \acl{DBM}\acused{DBM} (\acs{DBM}, see \autoref{sec:cmes}) --- and the statistical study of \citet{Winslow-2015} suggest that the deceleration of fast \acp{CME} due to their interaction with the slower ambient solar wind ceases before \SI{1}{\AU}, e.g. at distances between \SIrange[range-phrase={\,and\,}]{0.75}{0.85}{\AU}. \citet{Winslow-2015} have validated this hypothesis using data at Mercury and \citet{Wang-2005} have shown measurements from the \textit{Ulysses} spacecraft, mostly far beyond \SI{1}{\AU}, that show no significant deceleration, but observations at Mars have so far not been included in such investigations.

In the case of Earth and Mars, whose orbital periods are 365 and 687 days, respectively, an opposition occurs approximately every 780 days, or 2.1 (Earth) years --- but this number varies slightly due to the eccentricity of Mars's orbit. Since the \textit{Curiosity} rover's landing on Mars in August 2012, such oppositions have occurred four times: In April 2014, May 2016, July 2018, and October 2020. The following study will present the first statistical study of \acp{ICME} and the associated \acp{FD} during the first two of these opposition periods, and during oppositions of Mars with one of the two \ac{STEREO} spacecraft in 2012 and 2013. The \acp{FD} at the two locations can then be used to derive the \ac{ICME} propagation time between \SI{1}{\AU} and Mars. In the study, we compare the derived transit speed between \SI{1}{\AU} and Mars to the in situ measured velocity at \SI{1}{\AU} as well as the launch speed at the Sun to show that fast \acp{ICME} still decelerate beyond \SI{1}{\AU}. Comparisons with the WSA-ENLIL+Cone and \ac{DBM} models are also performed to investigate their accuracy for predicting \ac{ICME} arrival times at Mars.

\newpage

The following article is reproduced from \textcite{Forstner-2018} with permission from Journal of Geophysical Research: Space Physics, \copyright American Geophysical Union:\\

\pubcite{Forstner-2018}
\hfill Own contribution: 90\%

\newpage
\newcounter{includepdfpageJGREighteen}

\addtocounter{section}{1}
\setcounter{subsection}{1} 
\phantomsection
\addcontentsline{toc}{section}{\arabic{chapter}.\arabic{section} Using Forbush Decreases to Derive the Transit Time of ICMEs Propagating from 1 AU to Mars (Publication JGR--Space Physics 2018)}
%
\phantomsection
\addcontentsline{toc}{subsection}{\arabic{chapter}.\arabic{section}.\arabic{subsection} Introduction}
\label{sec:paper_forstner2018}
\includepdf[pages={1-2}, link, linkname=paper_forstner2018, scale=.95, pagecommand={\refstepcounter{includepdfpageJGREighteen}\label{paper_forstner2018.\theincludepdfpageJGREighteen}}]{publications/Forstner_et_al-2018-JGRSpace.pdf}
%
\addtocounter{subsection}{1} 
\phantomsection
\addcontentsline{toc}{subsection}{\arabic{chapter}.\arabic{section}.\arabic{subsection} Methods and Data}
\includepdf[pages={3-5}, link, linkname=paper_forstner2018, scale=.95, pagecommand={\refstepcounter{includepdfpageJGREighteen}\label{paper_forstner2018.\theincludepdfpageJGREighteen}}]{publications/Forstner_et_al-2018-JGRSpace.pdf}
%
\addtocounter{subsection}{1} 
\phantomsection
\addcontentsline{toc}{subsection}{\arabic{chapter}.\arabic{section}.\arabic{subsection} Results and Discussion}
\includepdf[pages={6-12}, link, linkname=paper_forstner2018, scale=.95, pagecommand={\refstepcounter{includepdfpageJGREighteen}\label{paper_forstner2018.\theincludepdfpageJGREighteen}}]{publications/Forstner_et_al-2018-JGRSpace.pdf}
%
\addtocounter{subsection}{1} 
\phantomsection
\addcontentsline{toc}{subsection}{\arabic{chapter}.\arabic{section}.\arabic{subsection} Conclusion}
\includepdf[pages={13}, link, linkname=paper_forstner2018, scale=.95, pagecommand={\refstepcounter{includepdfpageJGREighteen}\label{paper_forstner2018.\theincludepdfpageJGREighteen}}]{publications/Forstner_et_al-2018-JGRSpace.pdf}
%
\addtocounter{subsection}{1} 
\phantomsection
\addcontentsline{toc}{subsection}{\arabic{chapter}.\arabic{section}.\arabic{subsection} Appendix A: Cross-Correlation Analysis Plots for Each Event}
\includepdf[pages={14-16}, link, linkname=paper_forstner2018, scale=.95, pagecommand={\refstepcounter{includepdfpageJGREighteen}\label{paper_forstner2018.\theincludepdfpageJGREighteen}}]{publications/Forstner_et_al-2018-JGRSpace.pdf}
%
\addtocounter{subsection}{1} 
\phantomsection
\addcontentsline{toc}{subsection}{\arabic{chapter}.\arabic{section}.\arabic{subsection} References}
\includepdf[pages={17-18}, link, linkname=paper_forstner2018, scale=.95, pagecommand={\refstepcounter{includepdfpageJGREighteen}\label{paper_forstner2018.\theincludepdfpageJGREighteen}}]{publications/Forstner_et_al-2018-JGRSpace.pdf}

... wie viele weitere Events sind inzwischen im HELCATS-Katalog, die richtung Mars gingen? + blabla...`


\TODO{Summary of the publication}

\newpage

The following article is reproduced from \textcite{Forstner-2019} with permission from Space 
Weather, \copyright American Geophysical Union:\\

\pubcite{Forstner-2019}
\hfill Own contribution: 90\%

\includepdf[pages=-]{publications/Forstner_et_al-2019-Space_Weather.pdf}

\section{Outlook}