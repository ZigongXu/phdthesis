The goal of this thesis was to introduce observations at Mars, as well as \ac{SolO} into the framework of space weather observations in the inner heliosphere, and to gain a better understanding of the radial evolution of \acp{ICME} through the use of the \ac{FD} measurements available from the \ac{RAD} on Mars and from the \ac{HET} onboard \ac{SolO}.

In the first two publications in \autoref{chp:arrival_times}, we have assembled two catalogs of \ac{ICME} events that were associated with \acp{FD} at Mars, but were also observed from a second point --- either in situ at Earth or one of the \ac{STEREO} spacecraft near oppositions of these locations \citep{Forstner-2018}, or remotely from the \acused{STEREO}\ac{STEREO} \acp{HI} \citep{Forstner-2019}. These catalogs will serve as a useful resource in future studies, and should be continued through the inclusion of more recent events as well.
In the first article studying the events during opposition constellations, the cross-correlation function of the \ac{FD} measurements at \SI{1}{\AU} and Mars allowed us to directly derive the transit time for a statistical study. The comparison of the transit times with the in situ measured velocities at \SI{1}{\AU} allowed us to show for the first time that \acp{ICME} can continue to decelerate beyond the Earth orbit, and that this is effect depends on the relative velocity of the \ac{ICME} to the ambient solar wind. This confirms that theoretical models that are based on this relative velocity, such as the \acp{DBM}, are applicable even for these larger distances from the Sun. The arrival times were also compared to the results of the WSA-ENLIL+Cone \ac{MHD} model, and the typical deviations in the arrival time were comparable to those typically seen at other heliospheric locations.
In the second publication, we used the \acp{FD} at Mars to validate the \ac{HI}-based arrival time estimations. The performance of the single-spacecraft reconstruction methods is not particularly high, but in line with results at other locations previously compiled by \citet{Moestl-2017-HelcatsHSO}. Stereoscopic triangulation methods may enhance these results, but the loss of connection to the \ac{STEREO}-B spacecraft in 2014 (see \autoref{sec:stereohi}) and the focus of the \ac{HI} telescopes on the Earth-Sun line has prevented this for most Mars-directed events. In the future, data availability from the \acp{HI} onboard \ac{SolO} and \ac{PSP} may make the application of such methods feasible again, and a recovery of \ac{STEREO}-B in the coming years would of course be beneficial as well.

With the study presented in \autoref{chp:fd_properties}, we shifted away from the mere analysis of arrival times to the analysis of other \ac{FD} properties. Based on the catalog assembled in \citet{Forstner-2019}, we could reproduce a correlation of two \ac{FD} parameters with the \ac{RAD} measurements that was already known from previous studies at Earth, though the slope $A$ of this relation (not to be confused with the steepness of the \acp{FD} themselves) was different at Mars than at Earth. Through the consultation of analytical \ac{FD} models, we have found that this value $A$ is likely independent of the measured particle energy, and rather serves as a measure of the increase of the \ac{ICME} or sheath structure's size between the two planets. This was also made plausible by comparing the thereby obtained ratio of $A$ values with theoretical first-order approximations of the expected magnitude of this broadening. In future studies, this hypothesis should be further validated by analyzing sufficiently large samples \acp{FD} at other heliospheric locations in the same way.

In the two publications in \autoref{chp:september_event}, we have shown \ac{RAD} measurements of the most severe space weather event observed on the surface of Mars so far in September 2017, associated with a \ac{SEP} event as well as the merging of multiple \acused{CME}\acp{CME} en route to Mars, which caused an enormous \ac{FD} following the \ac{SEP} event. This serves as a case study of a complex space weather event seen at Earth and Mars, but should not be seen as a worst-case scenario, as the \ac{SEP} source did not have direct magnetic connection to Mars and the increased radiation dose during the event was almost compensated by the following \ac{FD}.

Finally, \autoref{chp:solo} introduced the measurement capabilities of the \ac{HET} onboard the \ac{SolO} mission, which launched in early 2020. While the measurement of \ac{FD} are not one of the main focuses of \ac{SolO} and instruments for the direct measurement of \ac{ICME} plasma and magnetic field are also available on the spacecraft, the first \ac{FD} observations shown here clearly show the high resolution with which such events can be captured by \ac{HET}. The close alignment of \ac{SolO} and Earth during this event also make it a good candidate for multispacecraft studies, and this was pursued in the publication with the application of the \acs{ForbMod} model to the \acp{FD} at \ac{SolO} and Earth. The reason for the disagreement of \acs{ForbMod} with the observations for this very slow \ac{CME} needs to be investigated in more detail in future studies, but our first investigations hinted that it may be due to interaction with the following \acl{SIR}.

In a nutshell, our work, both in statistical and case-study form, has highlighted different aspects of the propagation of \acp{ICME} in the inner heliosphere. Still, there are many open questions in this field, and future studies building upon our results will undoubtedly be important contributions to the better understanding of space weather events as well as the development of enhanced forecasting capabilities in the future.