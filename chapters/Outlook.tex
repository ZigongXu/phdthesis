The energetic particles in the heliosphere within the energy range of tens of Mev are of great importance, not only because of the abundant physics process they underly, but also due to their potential hazard to the human health, which is the requirement of space explorations.
Benefiting from two new instrumennts, \ac{LND} on board Chang'E-4 operating on the lunar surface and the \ac{HET} on board \ac{SolO} orbiting and approaching the Sun, and combined the multiple instrument observations,  we have a great oppurtunity to further our understand of these energetic particles.
Therefore, in this thesis, all three particle populations, \acp{SEP}, \acp{GCR} and \acp{ACR} are detailed investigated. The summary of the thesis and the outlooks is as follows.

%Amongest them, \ac{GCR}, ACR are from background and  SEP are the temporal increase. The particle full of different information of particle accerlerateion, injection, and propagtaion mechanismm.  
%Besides, the radiation hazard due those charge particles force us to study their time variation and spatial distribuions and the impact on the human body, which is the requirement of space explorations.


In the first publication presented in Chapter \ref{chp:LND_SEP}, we report the first \ac{SEP} event reaching the lunar far-side surface. This \ac{SEP}event have very weak intensity 
Our analysis consists both in-situ and remote-sensing data obtained from multiple instruments and perspectives. We first derived the proton integrated spectrum. Then we infered the release times of protons and electrons based on their clearly velocity dispersion. Based on results, We found an earlier injection of electrons than that of protons. 
Meanwhile, the solar eruption originating from a distant, solitary active region on the solar disk and a slowly moving \ac{CME} which is observed on the corongraph appear to be the most plausible source of these MeV energy particles in this event. We proposed potential transport mechanisms that could account for the traversal of protons and electrons within the lower coronal region and over the heliosphere, such as expanding \ac{CME}-driven shocks, irregular magnitic field lines, and even transport path diverging from the nominal Parker spiral field. Furthermore, in recent communications with Christina Cohen, the idea of the \ac{HCS} serving as the bridge between the active region and the \ac{L1} in this special case was brought up, even though the detailed transport mechanism remain unclear and need further investigation. Based on the above observations, we conclued that the protons and electrons behave differently in the release and transport process. Though we 

By analyzing various observations, we have provide plausibly solution in addressing the questions raised in the motivation section regarding the source of tens of MeV \acp{SEP} and their transport in the heliosphere. However, due to limited remote-sensing observation and single perspectives, we could not give a full picture of the event and determine the exactly roles that each potential factor play.  The recently study on the wide spread SEP from \citet{dresing202317, Kolhoff2021AA} indicates that the multiple observation and numberical simulation might help a lot on answering these questions.

In the second publications, the primary \ac{GCR} proton spectrum during 2019 - 2020 measured by \ac{LND} on the lunar surface is presented. This spectrum span the energy range from $\sim$ 10 to $\sim$ 400 MeV, including both the stopping particles and penetrating particles of \ac{LND}, though in the current stage, the uncertainty of penetrating data products is still higher than anticipated. The proton measurements from \acs{SOHO}/\acs{EPHIN} during the same periods in the energy range of 10 - 50 MeV are consistent with the \ac{LND} measurements. However, the GCR are higher than the previous solar activity minimum, reaching the historical record since the start of space age.  Moreover, we derived the intensities of 65 -76 MeV albedo protons on the lunar surface. The albedo protons are a subclass of the secondary particles which are generated after the interaction betweeen \ac{GCR} and lunar regolith. We found that the flux of albedo protons in this solar activity minimum is consistent with that in the previous solar acitivity minimum which was obtained by the \acs{LRO}/\acs{CRaTER} in 2009 - 2010. The study of albedo proton shed light into the lunar radiation environment and the seaching for the hydrogen on the lunar surface. 
%As a matter of fact, the \ac{LND} measurements can further define a albedo proton spectrum of several energy bins only if we have enough statistic. 
Meanwhile, the analysis of quite time spectra of protons,helium-4, helium-3, oxygen, carbon and iron from \citet{Mason-2021-SolOQuietTime} show how the ions spectra look like during the soalr minimum betweeen 0.5 - 1.0 au. The lower energy measurement are made by \acs{SIS} and the higher energy part are measured by \acs{HET}. These spectra is comprised of the GCR components, ACR components and lower energy turn-up spectrum which due to the impulsive \acp{SEP}. The ACR spectra is similar with that in the previous solar cycles. 

In the above two studies, we check the \ac{ACR}, \ac{GCR} in the heliosphere and on the lunar surface and secondary particles measured on the lunar surface.   We try answer the questions


%In particular, the super quiet time measurements show us how the particle spectra differ compare with normal quiet time. In the spectram GCR ACR and a turn-up spectrum 
%The helium turn-up 

%we did what, we found what, and the try to anwser which quesiton? Has you answer that? Any further expected from here?, what question are answer


In the last section of the thesis, we try to understand the distribution of the cosmic ray


The main topic of this thesis was to better understand the instrument and its new data around ten of Mev

The main goal is to look at the particles around ten of MeV using the new instrument and new time period

The data calibration and analysis are the main part of this thesis. 

Cross calibration of the new instruments:


In the end we try to give the answer to the instrumental question we ask in the motivation, how dose the new measurment look like, are they 


In the forth publication, we study the ACR measurements of Helium.

Let's re-visit the questions we 


We plan for the future study


\subsection*{Outlook}

By comparing the new measurement of \ac{SEP} and background cosmic ray, we conclude that new 



\subsubsection*{The relationship between SEPs and the radiation dose on the lunar surface}

LND and SOLO are two total new instruments.
LND have many many data and only few projects have been explored.
Below are several possilbe project that could be done with the LND data
1. Heavy ion spectra on the lunar surface. We only give the helium but the other heavy ion like carbon, nitrogen, and oxygen havent been explored. Data calibration need to be done( which is already finished in the past), 
2. Albedo particle more fine spectra
3. Connecting the observation dose and the model simulation - SEP and GCR
4. Discrepancy between LND and Crater.


\subsubsection*{The spatial gradient of ACRs and GCRs}
The spatial gradient of ACR oxygen and nitrogen


The spatial gradient of GCR helium


The spatial gradient of proton

% The goal of this thesis was to introduce observations at Mars as well as at \ac{SolO} into the framework of space weather observations in the inner heliosphere, and to gain a better understanding of the radial evolution of \acp{ICME}. This was achieved by making use of the \ac{FD} measurements available from the \ac{RAD} on Mars and from the \ac{HET} onboard \ac{SolO}.

% In the first two publications, shown in \autoref{chp:arrival_times}, we have assembled two catalogs of \ac{ICME} events that were associated with \acp{FD} at Mars, but were also observed from a second point --- either in situ during oppositions of Mars with Earth or one of the \acused{STEREO}\ac{STEREO} spacecraft \citep{Forstner-2018}, or remotely from the \ac{STEREO} \acp{HI} \citep{Forstner-2019}. These catalogs will serve as a useful resource for future studies and should be continued by including more recent events in the future.
% In the first article studying the events during opposition constellations, the cross-correlation function of the \ac{FD} measurements allowed us to directly derive the transit times from \SI{1}{\AU} to Mars for a statistical study. The comparison of these transit times with the in situ measured \ac{ICME} velocities at \SI{1}{\AU} allowed us to show for the first time that \acp{ICME} can continue to decelerate beyond the Earth orbit and that this effect depends on the \ac{ICME}'s velocity relative to the ambient solar wind. This confirms that theoretical models based on this relative velocity, such as the \ac{DBM}, are applicable even at larger distances from the Sun. The \ac{ICME} arrival times were also compared to the results of the WSA-ENLIL+Cone \ac{MHD} model, and we found that the mean deviations in the arrival time were comparable to those typically seen at other locations in the inner heliosphere.
% On the other hand, in the second publication, we used the \ac{FD} observations at Mars to validate the \ac{HI}-based arrival time estimations. The single-spacecraft reconstruction methods are not particularly precise, but their performance is in line with the results for other locations previously compiled by \citet{Moestl-2017-HelcatsHSO}. Stereoscopic triangulation methods may improve these results, but the loss of connection to the \ac{STEREO}-B spacecraft in 2014 (see \autoref{sec:stereohi}) and the focus of the \ac{HI} telescopes on the Earth-Sun line has prevented this for many Mars-directed events. In the future, data availability from the \ac{HI} instruments onboard \ac{SolO} and \ac{PSP} may make the application of such methods feasible again, and a recovery of \ac{STEREO}-B in the coming years as it comes back closer to Earth would of course be beneficial as well.

% With the study presented in \autoref{chp:fd_properties}, we shifted away from the mere analysis of arrival times to the investigation of other \ac{FD} properties. Based on the catalog assembled in \citet{Forstner-2019} as well as larger independent catalogs, we could reproduce a correlation of two \ac{FD} parameters with the \ac{RAD} measurements that was already known from previous studies at Earth, though the slope $A$ of this relation (not to be confused with the steepness of the \acp{FD} themselves) was different at Mars than at Earth. Through the consultation of analytical \ac{FD} models, we have found that this value $A$ is likely independent of the particle energies observed by the different instruments and that it rather serves as a measure of the increase of the \ac{ICME} or sheath structure's size between the two planets. This result was also supported by comparing the obtained ratio of $A$ values with theoretical first-order approximations of the expected magnitude of this broadening. In future studies, this hypothesis should be further validated by analyzing large samples of \acp{FD} at other heliospheric locations in the same way.

% In the two publications in \autoref{chp:september_event}, we have shown \ac{RAD} measurements of the severe space weather events observed on the surface of Mars in September 2017. These consisted of a \ac{SEP} event as well as the merging of multiple \acused{CME}\acp{CME} en route to Mars, which caused an enormous \ac{FD} following the \acp{SEP}. This serves as a case study of a complex space weather event seen at both Earth and Mars but should not be seen as a worst-case scenario, as the \ac{SEP} source did not have direct magnetic connection to Mars, and the increased radiation dose during the event was coincidentally almost compensated by the following large \ac{FD}.

% Finally, \autoref{chp:solo} introduced the measurement capabilities of the \ac{HET} onboard the \ac{SolO} mission, which launched in early 2020. While the measurement of \acp{FD} is not one of the main focuses of \ac{SolO} and instruments for the direct measurement of \ac{ICME} plasma and magnetic field are also available on the spacecraft, the first \ac{FD} observations we obtained clearly show the high resolution with which such events can be captured by \ac{HET}. The close alignment of \ac{SolO} and Earth during this event also makes it a suitable candidate for multispacecraft studies, and this was pursued in the publication with the application of the \acs{ForbMod} model to the \acp{FD} at \ac{SolO} and Earth. The reason for the disagreement of \acs{ForbMod} with the observations for this very slow \ac{CME} needs to be examined in more detail in the future, but our initial investigations presented in the article suggest that it may be due to interaction with the following \acl{SIR}.

% In a nutshell, this work, both in statistical and case-study form, has highlighted many different aspects of the propagation of \acp{ICME} in the inner heliosphere. Still, there are numerous open questions in this field, and future studies building upon our results will undoubtedly be important contributions to the better understanding of space weather events as well as the development of enhanced forecasting capabilities.