The main topic of this thesis was to better understand the instrument and its new data around ten of Mev

The main goal is to look at the particles around ten of MeV using the new instrument and new time period

The data calibration and analysis are the main part of this thesis. 

In the first publicaitons, we study the SEP

In the second publication, we study the GCR and albedo protons

In the third publication, we study the GCR protons


In the forth publication, we study the ACR measurements of Helium.

LND and SOLO are two total new instruments.
LND have many many data and only few projects have been explored.
Below are several possilbe project that could be done with the LND data
1. Heavy ion spectra on the lunar surface. We only give the helium but the other heavy ion like carbon, nitrogen, and oxygen havent been explored. Data calibration need to be done( which is already finished in the past), 
2. Albedo particle more fine spectra
3. Connecting the observation dose and the model simulation - SEP and GCR
4. Discrepancy between LND and Crater.

% The goal of this thesis was to introduce observations at Mars as well as at \ac{SolO} into the framework of space weather observations in the inner heliosphere, and to gain a better understanding of the radial evolution of \acp{ICME}. This was achieved by making use of the \ac{FD} measurements available from the \ac{RAD} on Mars and from the \ac{HET} onboard \ac{SolO}.

% In the first two publications, shown in \autoref{chp:arrival_times}, we have assembled two catalogs of \ac{ICME} events that were associated with \acp{FD} at Mars, but were also observed from a second point --- either in situ during oppositions of Mars with Earth or one of the \acused{STEREO}\ac{STEREO} spacecraft \citep{Forstner-2018}, or remotely from the \ac{STEREO} \acp{HI} \citep{Forstner-2019}. These catalogs will serve as a useful resource for future studies and should be continued by including more recent events in the future.
% In the first article studying the events during opposition constellations, the cross-correlation function of the \ac{FD} measurements allowed us to directly derive the transit times from \SI{1}{\AU} to Mars for a statistical study. The comparison of these transit times with the in situ measured \ac{ICME} velocities at \SI{1}{\AU} allowed us to show for the first time that \acp{ICME} can continue to decelerate beyond the Earth orbit and that this effect depends on the \ac{ICME}'s velocity relative to the ambient solar wind. This confirms that theoretical models based on this relative velocity, such as the \ac{DBM}, are applicable even at larger distances from the Sun. The \ac{ICME} arrival times were also compared to the results of the WSA-ENLIL+Cone \ac{MHD} model, and we found that the mean deviations in the arrival time were comparable to those typically seen at other locations in the inner heliosphere.
% On the other hand, in the second publication, we used the \ac{FD} observations at Mars to validate the \ac{HI}-based arrival time estimations. The single-spacecraft reconstruction methods are not particularly precise, but their performance is in line with the results for other locations previously compiled by \citet{Moestl-2017-HelcatsHSO}. Stereoscopic triangulation methods may improve these results, but the loss of connection to the \ac{STEREO}-B spacecraft in 2014 (see \autoref{sec:stereohi}) and the focus of the \ac{HI} telescopes on the Earth-Sun line has prevented this for many Mars-directed events. In the future, data availability from the \ac{HI} instruments onboard \ac{SolO} and \ac{PSP} may make the application of such methods feasible again, and a recovery of \ac{STEREO}-B in the coming years as it comes back closer to Earth would of course be beneficial as well.

% With the study presented in \autoref{chp:fd_properties}, we shifted away from the mere analysis of arrival times to the investigation of other \ac{FD} properties. Based on the catalog assembled in \citet{Forstner-2019} as well as larger independent catalogs, we could reproduce a correlation of two \ac{FD} parameters with the \ac{RAD} measurements that was already known from previous studies at Earth, though the slope $A$ of this relation (not to be confused with the steepness of the \acp{FD} themselves) was different at Mars than at Earth. Through the consultation of analytical \ac{FD} models, we have found that this value $A$ is likely independent of the particle energies observed by the different instruments and that it rather serves as a measure of the increase of the \ac{ICME} or sheath structure's size between the two planets. This result was also supported by comparing the obtained ratio of $A$ values with theoretical first-order approximations of the expected magnitude of this broadening. In future studies, this hypothesis should be further validated by analyzing large samples of \acp{FD} at other heliospheric locations in the same way.

% In the two publications in \autoref{chp:september_event}, we have shown \ac{RAD} measurements of the severe space weather events observed on the surface of Mars in September 2017. These consisted of a \ac{SEP} event as well as the merging of multiple \acused{CME}\acp{CME} en route to Mars, which caused an enormous \ac{FD} following the \acp{SEP}. This serves as a case study of a complex space weather event seen at both Earth and Mars but should not be seen as a worst-case scenario, as the \ac{SEP} source did not have direct magnetic connection to Mars, and the increased radiation dose during the event was coincidentally almost compensated by the following large \ac{FD}.

% Finally, \autoref{chp:solo} introduced the measurement capabilities of the \ac{HET} onboard the \ac{SolO} mission, which launched in early 2020. While the measurement of \acp{FD} is not one of the main focuses of \ac{SolO} and instruments for the direct measurement of \ac{ICME} plasma and magnetic field are also available on the spacecraft, the first \ac{FD} observations we obtained clearly show the high resolution with which such events can be captured by \ac{HET}. The close alignment of \ac{SolO} and Earth during this event also makes it a suitable candidate for multispacecraft studies, and this was pursued in the publication with the application of the \acs{ForbMod} model to the \acp{FD} at \ac{SolO} and Earth. The reason for the disagreement of \acs{ForbMod} with the observations for this very slow \ac{CME} needs to be examined in more detail in the future, but our initial investigations presented in the article suggest that it may be due to interaction with the following \acl{SIR}.

% In a nutshell, this work, both in statistical and case-study form, has highlighted many different aspects of the propagation of \acp{ICME} in the inner heliosphere. Still, there are numerous open questions in this field, and future studies building upon our results will undoubtedly be important contributions to the better understanding of space weather events as well as the development of enhanced forecasting capabilities.