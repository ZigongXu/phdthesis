

The main topic of this thesis is to better understand energetic particles measured by new instruments during the new solar cycle and advance our understanding of the instruments themselves. Energetic particles within the energy range of tens of MeV in the heliosphere are of great importance, not only because of the numerous physical processes they undergo but also due to their potential hazard to human health, which is the requirement for space exploration.

Benefiting from two new instruments, \ac{LND} on board Chang'E-4 operating on the lunar surface and the \ac{HET} on board \ac{SolO} orbiting and approaching the Sun, we have an excellent opportunity to improve our understanding of these energetic particles.
Therefore, in this thesis, all three particle populations in the energy range between a few MeV/nuc and a few hundred MeV/nuc, including \acp{SEP}, \acp{GCR}, and \acp{ACR} are thoroughly investigated. The summary of the thesis and the outlook are as follows.
%Amongest them, \ac{GCR}, ACR are from background and  SEP are the temporal increase. The particle full of different information of particle accerlerateion, injection, and propagtaion mechanismm.  
%Besides, the radiation hazard due those charge particles force us to study their time variation and spatial distribuions and the impact on the human body, which is the requirement of space explorations.


In the first publication presented in Chapter \ref{chp:LND_SEP}, we report the first \ac{SEP} event reaching the lunar far-side surface, during which the peak energy of protons is about 20 MeV.
Our analysis consists of in-situ and remote-sensing data from multiple instruments and perspectives. We first derive the proton integrated spectrum, which is comparable with measurements from other instruments. Then we infer the release times of protons and electrons based on their clear velocity dispersion. Based on the results, We find an earlier injection of electrons than protons. 
Meanwhile, the solar eruption originates from a distant but solitary active region on the solar disk. A slowly moving \ac{CME} is observed on the coronagraph. These two appear to be the most credible source of these MeV energy particles in this event. We propose potential transport mechanisms that could account for the traversal of protons and electrons within the lower coronal region and over the heliosphere, such as expanding \ac{CME}-driven shocks, irregular magnetic field lines, and even transport path diverging from the nominal Parker spiral field. Furthermore, in recent communications with Christina Cohen, the idea of the \ac{HCS} serving as the bridge between the active region and the \ac{L1} in this particular case was brought up, even though the detailed transport mechanism remains unclear and needs further investigation. Based on the above observations, we conclude that the protons and electrons behave differently in the release and transport process. 

After analyzing various observations, we have provided the most probable answers to the questions raised in the motivation section regarding the source of tens of MeV \acp{SEP} and their transport in the heliosphere. However, due to limited remote-sensing observation and in-situ measurements from single perspectives, we could not provide a complete picture of the event and determine the exact roles that each potential factor plays. Recently, \citet{dresing202317, Kolhoff2021AA} studied the widespread \ac{SEP} event and obtained a better global picture of various aspects of the event. I learned that the multiple observations and numerical simulations they utilized might help answer these questions. I should consider using these methods in my future research.



The second publication presents the primary \ac{GCR} proton spectrum measured by \ac{LND} on the lunar surface during 2019-2020. This spectrum spans the energy range from $\sim$ 10 to $\sim$ 400 MeV, including both the stopping and penetrating particles of \ac{LND}. However, at present, the uncertainty of penetrating data products is still more considerable than anticipated. More efforts are needed to improve these data products.

The proton measurements from \acs{SOHO}/\acs{EPHIN} during the same periods in the energy range of 10-50 MeV are consistent with the \ac{LND} measurements. Our measurements find that the intensities of lower energy \ac{GCR} protons are higher than those during the prior solar activity minimum, reaching a historical record since the start of the space age. 
Moreover, we derived the intensities of 65 -76 MeV albedo protons on the lunar surface. The albedo protons are a subclass of the secondary particles generated after the interaction between \ac{GCR} and lunar regolith. The flux of albedo protons in this solar activity minimum is consistent with that in the previous solar activity minimum, which was obtained by \acs{LRO}/\acs{CRaTER} between 2009 - 2010 on the lunar orbit. By calculating the intensity ratio of albedo to primary protons, we confirm that below approximately 50 MeV, the albedo protons are the predominant component of the total proton flux during quiet time on the lunar surface. The situation will change when \acp{SEP} arrive at the Moon.
The above studies of albedo proton shed light on the lunar radiation environment and the search for hydrogen content in the lunar regolith.





 
%As a matter of fact, the \ac{LND} measurements can further define a albedo proton spectrum of several energy bins only if we have enough statistic. 
Meanwhile, the analysis of quite time spectra of protons,helium-4, helium-3, oxygen, carbon and iron from \citet{Mason-2021-SolOQuietTime} show how the ions spectra look like during the soalr acticity minimum 24/25 betweeen 0.5 - 1.0 au. The lower energy measurement are made by \acs{SIS} and the higher energy part are measured by \acs{HET}. These spectra is comprised of the \ac{GCR} components, the \ac{ACR} components and lower energy turn-up spectrum which is due to impulsive \acp{SEP}. %Particularly, the \ac{ACR} spectra is similar to that in the prior solar cycles. 

In the above two studies, we analyze \acp{ACR}, lower enengy \acp{GCR} in the inner heliosphere and on the lunar surface, and albedo protons generated on the lunar surface. We examine and figure out the characteristics of these components during the solar activity minimum, and gain insights into the impact of solar modulation on the variability of cosmic rays. Additionally, we emphasize the importance of albedo protons as part of secondary particles in radiation environment of the lunar surface.



%In particular, the super quiet time measurements show us how the particle spectra differ compare with normal quiet time. In the spectram GCR ACR and a turn-up spectrum 
%The helium turn-up 

%we did what, we found what, and the try to anwser which quesiton? Has you answer that? Any further expected from here?, what question are answer

Chapter \ref{chp:ACR_Helium} is a forthcoming publication that focuses on heavy ions obtained from \ac{HET}. The goal is to understand the spatial distribution of cosmic rays in the inner heliosphere. As an ongoing project, we reported the data analysis in detail. Starting by cross-calibrating measurements from different instruments, we find that the spectra of heavy ions are consistent. Then, through a detailed analysis, we illustrate how the intensity of \ac{ACR} helium is affected by the long-term solar modulation and the temporal increase due to \acp{SEP}.
After properly considering the possible modulation effects, we derive the radial gradient of 10 - 50 MeV/nuc \ac{ACR} helium that are measured by \ac{SolO}/\ac{HET} between 0.3 and 1 au. The radial gradients are 28$\pm$ 8\%/au, 19$\pm$ 8\%/au, 10$\pm$ 11\%/au, 21$\pm$ 11\%/au for helium energies of 10 - 20 MeV/nuc, 20 - 30 MeV/nuc, 30 - 40 MeV/nuc, and 40 - 50 MeV/nuc respectively.
Such gradients are consistent with results from \ac{PSP}. We also provide the time variation of those gradients without further explanations, which will be the focus of the next step. In the future,  we plan to implement the same data analysis approach to \ac{ACR} oxygen in the energy range of 5 - 25 MeV/nuc and even the \ac{GCR} particles to draw a complete picture of the cosmic ray transport in the inner heliosphere.


In short, the excellent charged particle measurements obtained from \ac{LND} and \ac{HET} have vastly improved our understanding of the relationship between the Sun and the heliosphere and are of great help in preparing for future human exploration on the Moon and in deep space. We should take good care of these fragile instruments and fully employ the fruitful data they provide to expand our knowledge of the heliosphere.




% Several possible questions as we shown in that should be discussed and answered in the future studies.

 


% On the other hand, based on the further  instrument calibration based on the simulation and reprocess of data in the ground, we expand the energy coverage of the instrument and upgrade the scientific data products, like what have been done for \ac{EPHIN} \citep{kuehl2017PhDT}. \ac{LND} have witnessed in the last four years the change of the solar activities and the variation of the cosmic rays, will keep on providing the data to the commuity. 




% LND and SOLO are two total new instruments.
% LND have many many data and only few projects have been explored.
% Below are several possilbe project that could be done with the LND data
% 1
% 2. Albedo particle more fine spectra
% 3. Connecting the observation dose and the model simulation - SEP and GCR
% 4. Discrepancy between LND and Crater.


