

The main topic of this thesis is to better understand energetic particles measured by new instruments during the new solar cycle. 
% and advance our understanding of the instruments themselves. 
Since they contain critical information on particles' acceleration, injection, and propagation mechanisms in space, 
energetic particles within the energy range of tens of MeV in the heliosphere are of great importance. Besides, the even higher energy particles (hundreds of MeV) are a potential hazard to the health of astronaut who works in deep space, especially during the period without the protection of the spacecraft hull.
%Therefore, it is crucial to have a comprehensive understanding of those particles and the radiation they induce.
%numerous physical processes they undergo, but also due to their potential hazard to human health, which is the requirement for space exploration.

Benefiting from two new instruments, the \ac{LND} on board Chang'E-4 operating on the lunar surface and the \ac{HET} on board the \ac{SolO} orbiting and approaching the Sun, we have an excellent opportunity to improve our understanding of these energetic particles.
Therefore, in this thesis, all three particle populations in the energy range between a few MeV/nuc and a few hundred MeV/nuc, including \acp{SEP}, \acp{GCR}, and \acp{ACR} are thoroughly investigated. We summarize the key results we obtained and link them to the scientific questions we asked in the motivation section (Sec. \ref{sec:Motivation}). 
%The summary of the thesis and the outlook are as follows.
%Amongest them, \ac{GCR}, ACR are from background and  SEP are the temporal increase. The particle full of different information of particle accerlerateion, injection, and propagtaion mechanismm.  
%Besides, the radiation hazard due those charge particles force us to study their time variation and spatial distribuions and the impact on the human body, which is the requirement of space explorations.


In the first publication presented in Chapter \ref{chp:LND_SEP}, we try to answer two questions regarding the \acp{SEP}: "What is the source of those energetic particles?" (S1) and "How do those energetic particles arrive at distant longitudes?" (S2)
 
We report the first \ac{SEP} event measured on the lunar far-side surface, during which the peak energy of protons is about 20 MeV. This \ac{SEP} event persists for less than one day, and the time profiles of electrons and protons show clear velocity dispersion and are anisotropic during the start of the event.
We first derive the proton integrated spectrum, which is comparable with measurements from other \acs{L1} instruments. Then, we infer the release times of protons and electrons using the velocity dispersion method. We find an hour earlier injection of electrons than protons. Meanwhile, the flare originates from a distant but solitary active region on the solar disk. A slowly moving \ac{CME} appeared in the coronagraph of \acs{SOHO} after the eruption of the solar flare.
This intriguing event indicates the contradiction between the in-situ and remote-sensing observations. The in-situ measurements suggest that this event should be well-connected to the particle source. However, the remote-sensing data show that the active region is located at more than 100 degrees east of the magnetic footpoint of the \ac{LND}. The different release times of protons and electrons might indicate the different acceleration and transport mechanisms.
After excluding the possibility of local acceleration, the flare and the \ac{CME} driven shock appear to be the most credible source of these energetic particles in this event. 

We propose potential transport mechanisms that could account for the traversal of protons and electrons within the lower coronal region and over the heliosphere, such as expanding \ac{CME}-driven shocks, irregular magnetic field lines, and even transport path diverging from the nominal Parker spiral field. Furthermore, the idea of the \ac{HCS} serving as the transport bridge of energetic particles between the solar source and \acs{L1} in this special case was brought up \citep{Battarbee2018ApJ}, even though the detailed transport mechanism remains unclear and needs further investigation. 
%Based on the above observations, we conclude that the protons and electrons behave differently in the release and transport process. 
By analyzing multiple spacecraft/instrument observations, we discuss the source of tens of MeV protons and a few hundred keV electrons and their possible transport in the heliosphere. Our results indicate the necessity of using multiple spacecraft/instrument observations to study the \ac{SEP} events and the complexity of the \ac{SEP} events.

% However, due to limited remote-sensing observation and in-situ measurements from single perspectives, we could not provide a complete picture of the event and determine the exact roles that each potential factor plays. Recently, \citet{dresing202317, Kolhoff2021AA} studied the widespread \ac{SEP} event and obtained a global picture of . I learned that the multiple observations and numerical simulations they utilized might help answer these questions. I plan to consider these methods in my future research.

In addition to \acp{SEP}, \acp{GCR} and secondary particles, for example, albedo protons, are also essential components of the radiation environment on the lunar surface. Therefore, in the second publication, we present the \ac{GCR} proton spectrum and the intensity of albedo protons during 2019-2020. The studies of \acp{GCR} related to the question, "How do the \acp{ACR} and \ac{GCR} behave during the most recent solar activity minimum and the onset phase of solar cycle 25?" (C1); "What is the difference between the current solar cycle and the previous one?" (C2), and "How does solar modulation affect the intensity of cosmic rays?" (C3). Besides, the albedo proton measurements answer the question, "How do those secondary particles induced by those cosmic rays affect the radiation environment on the lunar surface?" (C5).

Both primary and albedo proton measurements are taken by \ac{LND} on the lunar surface for the first time. The primary \ac{GCR} proton spectrum spans the energy range from $\sim$ 10 to $\sim$ 400 MeV, including both the stopping and penetrating particles of \ac{LND}. However, at present, the uncertainty of the penetrating data products is still more significant than anticipated. More efforts are needed to improve these data products. The proton measurements from \acs{SOHO}/\acs{EPHIN} during the same periods in the energy range of 10-50 MeV are consistent with the \ac{LND} measurements. Our measurements find that the intensities of lower energy \ac{GCR} protons are higher than those during the prior solar activity minimum, reaching a historical record since the start of the space age, indicating the impact of solar modulation on the variability of cosmic rays.

Moreover, we obtained intensities of 65 - 76 MeV albedo protons on the lunar surface. The flux of albedo protons in this solar activity minimum is consistent with that in the previous solar activity minimum, obtained by \acs{LRO}/\acs{CRaTER} between 2009 - 2010 in the lunar orbit. By calculating the intensity ratio of albedo to primary protons, we confirm that below approximately 50 MeV, the albedo protons are the predominant component of the proton flux during quiet times on the lunar surface. 

%We emphasize the importance of albedo protons as part of the secondary particles in the radiation environment of the lunar surface due to their non-negligible contribution to the proton flux on the energy of few tens of MeV (C5). %The situation is expected to change with an increasing number of \acp{SEP} with increasing solar activity.


%As a matter of fact, the \ac{LND} measurements can further define a albedo proton spectrum of several energy bins only if we have enough statistics. 
Meanwhile, the analysis of the quite-time measurements of protons,helium-4, helium-3, oxygen, carbon, and iron from \citet{Mason-2021-SolOQuietTime} obtained the ion spectra during the solar activity minimum 24/25 between 0.5 - 1.0 au. The lower energy measurement is made by \acs{SolO}/\acs{SIS}, and the higher energy part is measured by \acs{HET}. These spectra are comprised of the \ac{GCR} components, the \ac{ACR} components, and the spectrum of the low energy particles, which could be the residual particles of the preceding impulsive \acs{SEP} events. These measurements show similar characteristics of particle spectra compared to previous solar activity minimum, providing the potential answer to question C2.


%In particular, the super quiet time measurements show us how the particle spectra differ compare with normal quiet time. In the spectram GCR ACR and a turn-up spectrum 
%The helium turn-up 

%we did what, we found what, and the try to anwser which quesiton? Has you answer that? Any further expected from here?, what question are answer

Chapter \ref{chp:ACR_Helium} is a forthcoming publication focusing on heavy ions obtained from \ac{HET} between 2020 and 2022. In particular, the focus is on the \ac{ACR} helium and their spatial distributions, which contain transport information of cosmic rays in the inner heliosphere. This part gives the answer to question, "How do the cosmic rays distribute within the inner heliosphere?" (C4). As an ongoing project, we only report the result of data analysis here without further explanation. Using cross-calibrating methods, we find that the spectra of heavy ions derived from different instruments are consistent. The intensity time profiles of \ac{ACR} helium are obtained after thoroughly removing short-term enhancement from \acp{SEP}. These profiles clearly show the decrease in the intensity of cosmic rays, which is caused by long-term solar modulation.
After properly considering the possible modulation effects, which are the most challenging part of our data analysis, we derive the radial gradients of 10 - 50 MeV/nuc \ac{ACR} helium that are measured by \ac{SolO}/\ac{HET} between 0.3 and 1 au. The radial gradients are 28$\pm$ 8\%/au, 19$\pm$ 8\%/au, 10$\pm$ 11\%/au, 21$\pm$ 11\%/au for helium energies of 10 - 20 MeV/nuc, 20 - 30 MeV/nuc, 30 - 40 MeV/nuc, and 40 - 50 MeV/nuc respectively. We also get consistent results from different methods.
Such gradients are consistent with results obtained from \ac{PSP} measurements. We also discuss the time variation of those gradients and their implication on the change of transport conditions as the increase of solar activities. The further explanation of these intriguing results will be the focus of the next step. In the future,  we plan to implement the same data analysis approach to \ac{ACR} oxygen in the energy range of 5 - 25 MeV/nuc and even the \ac{GCR} particles to draw a complete picture of the spatial gradients of cosmic rays in the inner heliosphere. 
We hope we can advance our current understanding of the transport of cosmic rays in the inner heliosphere.


In short, the excellent charged particle measurements obtained by \ac{LND} and \ac{HET} have improved our understanding of the relationship between the Sun and the heliosphere and are of great help in preparing for future human exploration on the Moon and in deep space. We should take good care of these fragile instruments and fully employ the fruitful data they provide to expand our knowledge of the heliosphere.



% Several possible questions as we shown in that should be discussed and answered in the future studies.
% On the other hand, based on the further  instrument calibration based on the simulation and reprocess of data in the ground, we expand the energy coverage of the instrument and upgrade the scientific data products, like what have been done for \ac{EPHIN} \citep{kuehl2017PhDT}. \ac{LND} have witnessed in the last four years the change of the solar activities and the variation of the cosmic rays, will keep on providing the data to the commuity. 
% LND and SOLO are two total new instruments.
% LND have many many data and only few projects have been explored.
% Below are several possilbe project that could be done with the LND data
% 1
% 2. Albedo particle more fine spectra
% 3. Connecting the observation dose and the model simulation - SEP and GCR
% 4. Discrepancy between LND and Crater.


