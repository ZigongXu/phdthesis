\section{Particles in the heliosphere}

Our heliosphere is an enormous region in space embedded in the \ac{ISM}, which encompasses all solar system planets and extends far beyond even the Kuiper belt. 
It is filled with a thin plasma consisting of various populations of particles, many of which originate from the Sun itself (\autoref{fig:heliospheric_energy_spectrum}). 
The most abundant population is the solar wind, a steady flow of plasma that exits the Sun radially and forms the \ac{IMF}. 
In the near-Earth space, the slow solar wind reaches typical speeds between \SIrange[range-phrase={\,and\,}]{300}{500}{\kilo\meter\per\second}.
As the Sun rotates, it creates an Archimedian spiral of magnetic field carried by the solar wind, the so-called Parker spiral.

\begin{figure}
    \centering
    \includegraphics[width=0.6\linewidth]{images/heliospheric_energy_spectrum}
    \caption[Spectra of oxygen ions in the near-Earth interplanetary space]{Typical spectra of oxygen ions in the near-Earth interplanetary space, showing the contributions from different populations. Other particle species show similarly shaped spectra when plotted as a function of energy/nucleon. (adapted from \url{http://helios.gsfc.nasa.gov/ace/gallery.html}).}
    \label{fig:heliospheric_energy_spectrum}
\end{figure}

However, our Sun is an active star, and thus, the flow of particles is not simply constant.
Coronal holes forming on the solar surface emit faster solar wind streams with speeds $\gtrsim \SI{600}{\kilo\meter\per\second}$, which interact with the neighboring streams of slower wind by forming \acp{SIR}, and, in the case that a coronal hole persists for multiple solar rotations, \acp{CIR}.
Furthermore, active regions can occasionally produce solar flares, sudden and intense emissions of light often associated with the release of high-energy ($\sim \si{\mega\electronvolt}$) \acp{SEP}.
These are believed to be powered by reconnection of magnetic field lines at the Sun, which leads to the release of energy and acceleration of particles.
They often also coincide with the eruption of plasma from the solar corona in the form of a \ac{CME} at speeds up to a few thousands of \si{\kilo\meter\per\second}.

At the high end of the energy spectrum (\autoref{fig:heliospheric_energy_spectrum}) up to \si{\giga\electronvolt} energies, we find \acp{GCR}, particles originating from outside the heliosphere and entering it with a quasi-constant and isotropic flux.
These high-energy particles can be modulated by the \ac{IMF} intensity and short-term magnetic structures, as well as by the atmospheres and magnetospheres of planets where it is measured.

% TODO: add citations.

\section{Coronal mass ejections}

\section{Forbush decreases}

\section{Motivation}

\ac{MSL}/\ac{RAD}