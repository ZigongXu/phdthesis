\section{Energetic particles in the heliosphere}
\label{sec:particles_heliosphere}


The heliosphere is a vast, bubble-like region in space that envelops the Sun. This region is moving with respect to the \ac{ISM} with a speed of about 25 km/s \citep{McComas2015ApJS}. The heliosphere can also be regarded as a plasma cavity. This cavity is created by the Sun and is governed by the solar wind and its embedding magnetic field. It is filled by particles with various origins. The particle populations are identified from Fig.\ref{Fig:Oxygen_spectra_heliosphere} which is adapted from \citet{Mewaldt-2001}. Based on the accumulated measurements of oxygen by the \ac{ACE} between 1997 and 2000 at 1 au, the oxygen fluence spectrum which spans over more than seven orders of magnitude from keV/nuc to GeV/nuc provides clear insight into the lower energy particles including the slow solar wind, the fast solar wind, the suprathermal tails, and high energetic particles composed of \acp{SEP}, \acp{ACR}, and the extremely high energetic \acp{GCR}. 

%Among them \acp{GCR} originate from distant sources outside the solar system, while \acp{ACR} sources are located near the boundary of the heliosphere. The remaining energetic particles are accelerated and generated inside of the heliosphere at the multiple locations, including the solar surface, interpslanetary space and even the planets, for instance Jupiter.

\begin{figure}[!htb]
	% moving the figure to the second pages
	\centering
	\includegraphics[width = 0.8\textwidth]{images/heliospheric_particle_spectra_color.png}
	\caption[Energy spectra of oxygen ions in near-Earth space]{The typical oxygen spectra in the interplanetary space near Earth, indicating the contributions of different particle populations, particularly in the energy range between few MeV/nuc and few hundreds MeV/nuc (green shaded region), where \acs{SEP}, \acs{ACR} and \acs{GCR} coexist. The spectra of other particle species such as, helium and protons have a similar shape but a different flux level in the corresponding energy regime. The figure is adapted from \citet{Mewaldt-2001}}
	\label{Fig:Oxygen_spectra_heliosphere}
\end{figure}


The solar wind is a stream of charged particles released from the solar corona, the upper atmosphere of the Sun. This plasma consists of mainly protons and electrons that continously flow outward and expand to about $\sim$ 100 au (depending on the direction and the phase of solar activity cycle). The typical energy range of the solar wind is between 0.5 keV and 4.5 keV. Depending on the locations that produce the solar wind, the speed and density of the solar wind might be different. For instance the fast solar wind with a typical speed between 500 and 800 kilometers per second is emitted from the coronal holes which are funnel-like regions of open field lines in the magnetic field and usually appear at the north and south pole of the Sun \citep{Sakao2007, Tu2005, hundhausen1968state}. Therefore, the fast solar wind dominates the high latitude regions during the solar activity minimum. On the other hand, the slow solar wind is observed to have a velocity of about 300 - 500 kilometers per second and is believed to originate from the equatorial region. Some may also come from the boundaris of coronal holes. The slow solar wind is more likely to be observed in the low latitude regions.


%plasma embedding with magnetic field

Suprathermal particles are ions and electrons that move about two to hundreds times faster than the solar wind particles. In the spectrum shown in Fig.\ref{Fig:Oxygen_spectra_heliosphere}, the suprathermal particles are beyond the tails of the fast solar wind and are the dominant particle population between few keV to few Mev. The source of the suprathermal particles might be the accelerated solar wind and the remanents of previous solar eruptions and \ac{SEP} events \citep{Gloeckler1995SSRv}. Suprathermal particles are suspected to play an important role as seed particles for \ac{SEP} events \citep{Kahler2019ApJ}.
%Those particles play an important role in contributing seed particles for \ac{SEP} events.

Above the energy of suprathermal particles is the energy range that we are interested in this thesis, especially the energy range between few MeV/nuc and few hundred MeV/nuc where the dominant particles are \acp{SEP} (not limited to this energy range), \acp{ACR} (up to $\sim$ 100 MeV/nuc) and lower energy \acp{GCR}. The measurements we used in this study are from this energy range.

\acp{SEP} are high-energetic particles with energies ranging from few keV up to $\sim$ GeV. They are emitted from the Sun and accelerated by solar flares and \ac{CME}-driven shocks. \acs{SEP} events are intermittent, short term, and normally intensive, compared with cosmic rays \citep{Reames1999}. Different types of \acs{SEP} events persist for different time scales from few hours to few days. %Such high energy particles are one of the major threats to the space environment and human activities in space.
%also particle from solar, accelerated by different mechanism. The enery range of \ac{SEP} are quite broad, especially depending the on where the measurement carried on. Recently \ac{SOLO} and \ac{PSP} frequently measure the hundreds keV \ac{SEP}.

\acp{ACR} are believed to be the high energy interstellar pick-up ions \citep{Giacalone2022SSRv}. Pick-up ions are born from neutral interstellar atoms getting ionized by solar UV radiation and charge exchange process with solar wind after the neutral atoms flow into the heliosphere. Those ionized particles are then carried by the expanding solar wind to the outer heliosphere, where they are accelerated somewhere near the termination shock to higher energy and then redirected inwards \citep{russell2016space}. The exact accelerated site is still unclear. The typcial \ac{ACR} species that have been observed are protons, helium, oxygen, nitrogen, neon and argon. 

\acp{GCR} are fully ionized particles that are accelerated at so-called supernova remnants \citep{Blasi2013AARv2013} outside of the solar system. Those energetic particles bombard Earth constantly. The complete spectrum of \acp{GCR} cover the energy from typical few MeV \citep{Potgieter2013LRSP} to TeV which is even larger than the energy range in Fig.~\ref{Fig:Oxygen_spectra_heliosphere}. \acp{GCR} are comprised of about 99\% ions and 1\% electrons.

After entering the heliosphere, the transport of both \acp{ACR} and \acp{GCR} are controled by the \ac{HMF}, hence \ac{ACR}'s and \ac{GCR}'s temporal variaton is highly related to the solar activity and so-called solar modulation. More details of the solar modulation will be discussed in Section \ref{Sec:GCR}.


% LND and SOLO/EPD are new instruments;

% In the helioshphere, 

% Questions:


% 0. Cross calibrate the new data from new instrument; how is the performance of the new instrument? Are they good enough? - To answer this question
% 1. How the widespread SEP generated



\section{Motivation}
\label{sec:Motivation}

So, why do we choose energetic particles in the tens to hundreds of MeV/nuc range as the central topic of this thesis? First, it is worth noting that charged particles within this energy range exhibit complicated properties and consist of three distinct particle populations, varying depending on time, energy range, and particle species. Disentangling the specific particle types within this energy range is exceptionally challenging, in particular when the Sun is active.
By investigating these charged particles, we have the opportunity to gain insights into the origin, acceleration, and transport mechanisms of \acp{SEP}, \acp{ACR}, and \acp{GCR}. In this thesis, we focus on the questions that we bring up below.

Furthermore, as space exploration continues to advance, energetic particles are recognized as one of the most dangerous radiations in space, especially those of hundreds of MeV energy. They pose potential threats to astronaut health and have large possiblity causing direct radiation damage such as nausea and vomiting and after-effect like cancer. They could also fail the functioning of electronic devices \citep{Reames2021LNP,  mckenna2015recommendations, armstrong2014strategies}. The seconday particles generated on the lunar surface from the interaction between cosmic rays and the lunar regolith can also reach such high energy and contribute significant radiation dose rate on the lunar surface \citep{Schwadron2016Icarus}, which are particularly important for manned missions to the Moon. Therefore, understanding the variations and distribution of those energetic particles or even higher energy particles in space and on lunar surface is particularly important for future human space exploration.

Lastly, another compelling reason for the investigation of the particles of interest here is the data from two new instruments. The instruments are the \ac{LND} on board Chang`E - 4 and the \ac{HET} on board the \ac{SolO}, which were developed by the extraterresrial physics department at Kiel University and measure charged particles within this energy range. Chang`E-4 is the first mission that was ever launched to the far-side surface of the Moon. It serves as a monitor of the radiation environment and a charged particle telescope. Meanwhile, \ac{HET} provides an excellent opportunity to measure energetic particles within one au. We have the opportunity to study many questions using the new measurement of tens of MeV/nuc energetic particles \citep{Wimmer2020SSRv,RodriguezPacheco-2019-EPD}, such as the questions we ask below.
%Analyzing these new measurements allows for a 
%reassessment of the questions related to tens of MeV/nuc energetic particles in the new solar cycle . 
Consequently, those studies can advance our understanding of particles in space and the radiation environment on the lunar surfaces.


% such as what is the origin of those particles? How are those particles accelerated? How are those particles transported in the heliosphere? After those particles arrive on the surface of the Earth or the Moon, how does the particle affect the life and activities on the surface of the Earth or the Moon?

There are numerous intriguing questions regarding those energetic particles within the heliosphere. In this thesis, we mainly focus on the following questions:

% \subsection*{Instrumental questions}

% How does the new data from \ac{LND} and \ac{HET} look, and how can we calibrate and verify new instruments using charged particles with different input spectra?
\begin{tcolorbox}[colback=blue!5!white,colframe=blue!75!black,title=Scientific questions - \acp{SEP}]
	S1: What is the source of those tens of MeV particles?  \\
	\hfill

	S2: How do those tens of MeV particles arrive at distant longitudes?
	
\end{tcolorbox}
	

%Therefore, we try to answer the above questions by analyzing the first \ac{SEP} events observed on \ac{LND} and other instruments. 
%are mainly discussed in the first publication of the thesis, where we study the first \ac{SEP} event observed, though we only have limited observation of this \ac{SEP}

\begin{tcolorbox}[colback=blue!5!white,colframe=blue!75!black,title=Scientific questions - cosmic rays]
	C1: How do the \acp{ACR} and \ac{GCR} behave during the most recent solar activity minimum and the onset phase of solar cycle 25? \\ \hfill

	C2: What is the difference between the current solar cycle and the previous one?\\ 
	\hfill

	C3: How does solar modulation affect the intensity of cosmic rays?  \\
	\hfill

	C4: How do the cosmic rays distribute within the inner heliosphere?  \\
	\hfill

C5: How do those secondary particles induced by those cosmic rays affect the radiation environment on the lunar surface?
\end{tcolorbox}
	

\acp{SEP} are the product of multiple processes that fill the heliosphere and are observed at different locations. With limited satellite observations from few locations in space, it is hard to figure out different processes involved in the generation and transport of \acp{SEP}. In the new space era, more and more spacecraft and instruments are deployed in space at different locations. Those significant observations allow us to study the \acp{SEP} from different perspectives. A case study of \acp{SEP} is given in section \ref{chp:LND_SEP} of this thesis.

Meanwhile, the recent solar activity minimum ended in 2020, before the starting of the new \ac{SC} 25. Notably, this solar cycle has exhibited unusual properties compared to the previous solar cycle. For example, observations have indicated historically high levels of \ac{GCR} flux during this period, exceeding the space-age records \citep{Fu2021ApJS, Xu2022FrASS}, but \ac{ACR} intensities have not reached the same record-setting levels \citep{Strauss2023ApJ}. 
In addition, solar activities increased rapidly during the onset of \ac{SC} 25, suggesting that this solar cycle could be the strongest since records began \citep{Nagovitsyn2023SoPh}. The peak of the \ac{SC} might arrive one year earlier than anticipated \citep{Prasad2023SoPh,McIntosh2020SoPh}. 
We discuss those questions regarding the cosmic rays during the new solar cycle in \citet{Xu2022FrASS,Mason-2021-SolOQuietTime} and the fourth part below, where we report the observations of \ac{ACR} helium.



The structure of the thesis is as follows: After the introduction in this section, we give the observational and theoretical background of \acp{SEP}, \acp{GCR} and \acp{ACR} in the heliosphere in Sec.~\ref{chp:background}. In Sec.~\ref{chp:instruments}, we briefly introduce the two new instruments - \ac{LND} and \ac{SolO}/\ac{HET} - and the data that we use in this thesis. In the next chapters, we report our findings of the first \ac{SEP} observed on the lunar far-side surface. In Sec.~\ref{chp:LND_GCR_albedo} we report \acp{GCR} and secondary protons measured on the lunar surface. In Sec.~\ref{chp:SOLO_Quite_time} we report the latest quiet time spectra of energetic particles measured by \ac{SolO} in the inner heliosphere. And in Sec.\ref{chp:ACR_Helium}, we report the \ac{ACR} helium radial gradients based on the newest measurements from \ac{SolO}/\ac{HET}. The summaries and outlooks are given in the last section.
In particular, more detailed information of the \ac{LND} is provided in the Appendix ~\ref{chp:LNDinstrument}.


% We might ask: Where are those particles come from? How these particle generated? How those particle arrived Earth or spread through the heliosphere? 
% How these particle affect our life? 
% 	How to protect ourself from those particle?
% Origin, acceleration, transport

% With new measurement we try to give the answer to the following questions:

% Origin:
% Where are \ac{SEP} from?

% Acceleration:
% No

% Transport:
% How \ac{SEP} spread in the heliosphere? and how the solar modulation affect the cosmic rays?

% Impact:
% How the energetic particles affect the life and activities on the surface of Moon?

% New instruments:
% Are those instrument perform well? How cross calibrate between the new instrument and the old one tell us?

% We know SEP from sun, but do we really know which source is respoonsible for those energetic particles? Flare or CME?  

% How the ACR transport

% Common question of energetic particles




% What we try to solved:

% The tranport of energetic particles in the heliosphere, including \ac{SEP}, \ac{ACR}

% \todo[options]{
% 	- motivate the topic of your thesis, explain why this is something interesting to work on 
% - discuss what questions your thesis aims to address
% - provide an overview on how these questions have been addressed in the preexisting literature, and explain how and why you work is necessary/ relevant to contribute to this. 
% - optioannly give an overview on the structure of the thesis.
% }
 

% questions:
% motivation to study radiation envirnoments:

% overview on the structure of the thesis
% the following we already talked about on Friday:

% From my point of view, Chapter 1 here is not the "introductiion" but the theoretical background.

% From my point of view, the introduction should 
% - motivate the topic of your thesis, explain why this is something interesting to work on 
% - discuss what questions your thesis aims to address
% - provide an overview on how these questions have been addressed in the preexisting literature, and explain how and why you work is necessary/ relevant to contribute to this. 
% - optioannly give an overview on the structure of the thesis. 

% A dateiled discussion/ overview on the thereitical background should then (from my point of view come in the next chapter)


% What is the motivation to study radiation envirnoments at all? (this should be included/ pointed out as part of the motivation in the introduction)

% Why are the three points ordered in this way?

% From the papers I would expect that: understanding ACR transport (which is imprortant because of ...) is also part of the motivation for your thesis ;

% If you can be more specifiv in the motivation, the better:
% what is the motivation for your research questions? 

% The interesting new solar cycle, the exiciting new missions and the opportunities for multispacecraft observations are all relevant, bu they don;t yet answer:
% - why are these missions (including SOlO and LND) so exciting?
% - what new questions can multipoint measurements address? which of these are you interested in? (what other questions are other people working on with multipoint observations?)
% - what makes Mars and Moon parsticularly interesting radiation environments?
% Future colony, human activity, astronauts

% -why is it interesting that the current solar cycle is nusual (and how can yout thesis contribute to that)?
%  The unusual solar cycle 



% questions:



% The inspiration of this thesis arises from the following three aspects:
% \begin{itemize}
% 	\item New missions and new measurements:
% 	%Over the last few years, several thrilling missions have been successfully launched after extensive preparation, such as \ac{PSP}, \ac{SolO}, \ac{Bepi}, lunar mission like the Chang'E series mission, \ac{ESA}'s Jupiter Icy Moons Explorer(Juice),
% 	% and Chinese missions like CHASE and ASO-S. 
% 	In this thesis, w
% 	Once we have the new data from the new instrument, the most fundemental question is how the data looks like in this new instrument.
% 	%Are they useful? Any new insight shed into the community: validate the instrument performance;
% 	\item New solar cycle:  On the other hand, after the solar activity minimum, the increasing solar eruptions and \ac{SEP} events provide researchers more oppurtunities to study solar activities and their impact on the Earth and planet.
% 	%Question: How the GCR modulated during the solar minimum? what is the drift and diffusion of ACR looks like in the new cycle?

% 	\item  %Question: How the wide spread event look like? Any discrepancy between the observation from SOLO and other instruments.
	
% \end{itemize}



% The exploration of space has witnessed a surge in intensity, with an increasing number of countries aspiring to venture into this domain. Noteworthy examples include NASA's initiation of the Artemis mission, which aims to return to the Moon by 2024. Similarly, China has unveiled its plans to establish a lunar base on the lunar surface by the 2030s, while the European Space Agency (ESA) has also embarked on a lunar lander mission. Most recently, a Japanese lunar lander mission was launched; however, it regrettably encountered failure.

% Under these circumstances, the study of solar energetic particles (SEPs) assumes greater significance. SEPs pose a significant radiation hazard for future human exploration on the lunar surface. The most hazardous SEP events have the potential to induce radiation increases of substantial magnitude.

% SEP events directed towards Earth can become an issue of space weather and
% very energetic events can cause a so-called Ground Level Enhancement (GLE).
% This means that the radiation level on the ground increases which can be seen in
% neutron monitor measurements.