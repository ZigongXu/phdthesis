\section{Energetic particles in the heliosphere}
\label{sec:particles_heliosphere}


Our heliosphere is a vast region in space embedded in the \ac{ISM}, which encompasses all solar system planets and extends far beyond even the Kuiper belt. 
It is filled with a thin plasma consisting of various populations of particles, many of which originate from the Sun itself. These populations can be identified in \autoref{fig:heliospheric_energy_spectrum} \citep[based on measurements by][]{Mewaldt-2001}, shown as an energy spectrum that extends over more than 7 orders of 


Introduce all the particles population from lower energy to the higher energy, each population has one paragraph, according to the plot you add, and maybe one plot showed the position of multiple process, could be the plot of blue or the plot in Paddy's thesis

As shown in Fig.\ref{Fig:heliospheric_energy_spectrum}, from the lower energy end to the higher energy end, we could find several different component. Solar wind, suprathermal particle, SEP, ACR, GCR. 
The shadow region indicating the energy range between few MeV to few hundreds MeV, where SEP, ACR and GCR are co-exist. 


SEP are the particle from solar, accelerated by different mechanism.
The enery range of SEP are quite broad, especially depending the on where the measurement carried on. Recently SOLO and PSP frequently measure the hundreds keV SEP.



GCR, fully ionized particlesoriginating outside the heliosphere ,accelerated ar supernova remnants [Blasi 2013], covering the energy from typical 1MeV ([Potgieter 2023 LRSP] to ZeV, 90\% of hydrogen, the remain 10\% consist of heavier ions ( percent \%), electron, positron and antiproton (?\%)
The flux , outside the heliosphere,  is isotropic distributed in space and nearly constant in time.  It is believe the source 

ACR are the high energy interstellar particles, accelerated, ionizing,. 


Both ACR and GCR's temporal variaton is highly related with the solar activities and the so-called solar modulation, which are periodcally change in 11 and 22 year period. 

\begin{figure}
	\centering
	\includegraphics[width = 0.6\textwidth]{images/heliospheric_particle_spectra_color.png}
	\caption[Energy spectra of oxygen ions in the near Earth space]{The typical oxygen spectra in the interplanetary space near the Earth, indicating the contributions of different populations, especially in the energy range between few MeV\/nuc and few hundreds MeV\/nuc, where \acs{SEP}, \acs{ACR} and \acs{GCR} both exist. The spectra of other particles species for instance, helium and proton, have the similar shape but different flux level on corresponding energy region. The figure is adapted from \cite{Mewaldt-2001}}
	\label{Fig:Oxygen_spectra_heliosphere}
\end{figure}

\begin{itemize}
	\item The few tens of MeV energy range is an very important energy range for the heliosphere. In this part the SEP, ACR, lower GCR are bothe there. Therefore, more attention required here
	\item 
\end{itemize}

\section{SEP}

discovery of SEP and history of SEP
- first SEP discovery and the 
\missingfigure[options]{First SEP plot, GLE}
-  Source of SEP at 1980 - flare
 - evidence 1
 - evidence 2
 - most represented result 1
flare math
Two type of SEP and how the source of SEP evolved

Wide spread SEP
	- 
Multi-instrument observation of the SE
The problem of SEP studies



\section{Galactic cosmic rays} ( 1500 words are enough, 50 citation)

\begin{figure}
	\centering
	\includegraphics[width = 0.5\textwidth]{images/oxygen_cosmic-ray spectrum.png}
	
	\caption{The cosmic spectra of all particles and ACR oxygens observed at 1 AU. This figure is from Giacalone 2021, 2012, and also originally from Jokipii 1990.
	The spectrum is displayed in more than 15 orders magnitude on the energy scale and about 30 orders of magnitude on the intensity scale, much broader than Fig.\ref{Fig:Oxygen_spectra_heliosphere}.}
	\label{Fig:Oxygen_spectra_cosmic_ray}
\end{figure}

The first discovery of the galactic cosmic rays was made by the Victor Hess in 1912 in a ballon experiement, in order to find the source of ionizing radiation in the atmosphere. In the experiment, as the ballon climb-up, he measured the radiation in earth atmosphere in a balloon experiement, finding an enhancement of the ionizing radiation in an electroscope.  


Cosmic rays, first discovered by physicist Victor Hess during a daring hot-air balloon flight in 1912, are subatomic particles moving at nearly the speed of light. These particles, primarily composed of protons, helium nuclei, and high-energy electrons, originate from a variety of sources both inside and outside our galaxy. GCRs, as the name suggests, are predominantly sourced from within the Milky Way.



GCRs consist of wide range of energetic particles,  which mostly are dominated by the proton, about 89\%  and the remain share are 10\% helium and a small portal of heavier ions (1\%), electron, positron and antiproton. The dominated energy of gcr, as shown in the Fig.\ref{Fig:Oxygen_spectra_heliosphere} is above 100MeV/nuc. Below that, the other component are more common than GCR and it is hard to seperated those particles.

Figure \ref{Fig:Oxyge_spectra_cosmic_ray} are the full spectrum of the all cosmic particles observed at 1 AU which includes the ACR and GCR components.  The GCR


Currently it is believed that those GCR are main originated from the supernova explosion and derived the energy from the shock waves which is genereate from the explosion. When the shock wave travel through the surrounding interstellar gas, the kinetic energy of shock are tranfer to the  (neutral gas?) by the (Fermi-acceleration, ciataion and the acceleration process), Utilimatly, causing the galactic cosmic ray up to 10$^12$ eV.

Because cosmic rays are fully charged, they are deflected by the magnetic field when they propagation in the interstellar space after speeding up. The directions of those particle are normalized by the strong magnetic field. Hence when they arrived at the local bubble of solar system, we obtained an nearly isotropic and constant intensity profile [citaion of the LSTM,]


To model the solar modulation on the particle transportation of GCR spectra, an input particle spectra need to be specifield, which is the so-call LSTM [ citaion of LSTM]. LSTM is the modulation boundary and will be modulatined as the change of the position, energy, and time after those particle diffusion into the heliosphere. Such a spectra have been observed by the voyeger after then cross the boundary of solar system and enter the interstellar medium


A paragraph of local intersteallar spectra ?


After enter the heliosphere, those high energetic particle are modulated the solar wind and its embedding magnetic field  which change in a 11 year or 22 year period,
The relavant process of the solar modulations could be described by a basic transport equation (TPE) which is first derived by Parker (1965). The same equations was also derived by Gleeson and Axford (1967) in the more rigorous ways. This equation is based on the motion of charged and particle in the high frequently changed magnetic field and averages over the pitch angle of particle moving in the magnetic field. The precondition of this equation is the reasonable assumption of the isotropically distributed GCRs. The TPE give the phase-space distribution function, $f$ as the function of positions, time  and momentum magnetitude. In Potgieter (2013), the helispheric TPE, based on Parker (1965) is rewritten in the following form:

	\begin{equation}
		\underbrace{\frac{\partial f}{\partial t}}_{a} = - ( \underbrace{\boldsymbol{V}}_{b} + \underbrace{\langle v_d \rangle }_{b}) \cdot \nabla f + \underbrace{\nabla \cdot (\boldsymbol{K_s \cdot \nabla f})}_{d} + \underbrace{\frac{1}{3}(\nabla \cdot \boldsymbol{V}) \frac{\partial f}{\partial ln P}}_{d}
		\label{Eq:Transportation_equation}
	\end{equation}

where $f(r, P, t)$ is the cosmic ray distribution as the function of the time t, particle rigidity P and 3-dimension position in the space. Compared with the $\sim$ 11 years solar cycle, the periodcally solar rotation ($\sim$ 27 days) and  the time of the solar wind traveling to the edge of helipshere ($\sim$ 1 years) are short-term variation and can be neglected. Hence the steady-state solution with  $\frac{\partial f}{\partial t} = 0$ (part a of Eq.\ref{Eq:Transportation_equation}) is a reasonable assumption and also considered. Terms in the right parts include four effects that are used to describe the variation of the cosmic rays: (b) convection due to the solar wind velocity $\boldsymbol{V}$; (c) drift effects caused by the gradient and curvature of the large-scale \ac{HMF}, which is estimated by a 3D Archimedean spiral (Parker 1978), $\langle v_d \rangle$ represents the averaged drift velocity; (d) diffusion effects caused by the turbulent mangetic field, with the $\boldsymbol{K}_s$ the symmetrical diffusion tenser; (e)adiabatic energy change and deceleration due to the expansion of the solar wind. 

Since TPE is a high non-linear partial differential equation, only a simplified solution of the GCR spectra is the Force-field Solution (FFS), which is first derived by Gleeson and Axford [1967, 1968b]. Later, a reasonable GCR spectra of the particle with energy above 150 MeV were given by Gleeson and Urch 1973.

With the development of computer technque and numerical studies, simulation are becoming more and more important in studying the tranporation and solar modulation of the cosmic rays. [Jokippi and Kopriva 1979, Le Roux and Potgieter 1995, Manuel et al 2011, and Potgieter 2013]. 
Several model like BON14, 2020, CREME 96 and [Find the paper give the name]. 



	
2 Temperal and spatial variation

\begin{figure}
	\centering
	\includegraphics[width = 0.6\textwidth]{images/Solar_modulation.png}
	\caption{The solar modulation potential \cite{Usoskin 2011}, data are downloaded from \url{https://cosmicrays.oulu.fi/phi/phi.html}, and the monthly averaged sunspot number from Solar Influences Data analysis Center (SIDC), Royal Observatory of Belgium, Brussels, \url{https://www.sidc.be/silso/datafiles}}
	\label{Fig:Solar_modulation}
\end{figure}


3.  temperal variation and GCR solar modulation 
	 The temperal variation comprise of 
	- When the arrived on the local bubble of Solar system- the solar modulation dominate the transport of the particle - General say about 


4. 	GCR spatial variation
	What is ?
	Why?
	Who first study this topic 



\subsection{ACR}
 - discovery 1970's found the ACR, by analysie the spectrum of GCR, increase of the flux in the spectrum, 
 Explain Figure \ref{Fig:Oxygen_spectra_cosmic_ray}, based on the description of Giacalone 2021

 Soon after the discovery, 
 it is believe the ACRs are the high energy interstellare pick up ions which are accelerated in termination shock of the heliosphere. 
 describe this 

 ACR charges states which is key that we can distinguish the ACR from GCR. 

 The possible acceleration mechanisam of the ACR include ???? (citaion)

ACR enter the heliosphere and diffuse in the heliosphere, interacting with solar wind, and magnetic field. The transport process of ACR is similar to the GCR, which could both describe by the equation \ref{Eq:TPE}, which contain four main tranposrt effect: 

Similar to GCR, ACR flux also have solar cycle dependence, and transportation process is also the same as GCR, but the different energy particles.
with charge sign different
,

Previous study and observation of the ACR gradient
	Radial and latitudian ACR gradient outside of 1AU

	Below 1 AU, New observation of ACR by PSP, the study of ACR gradient in the innner heliosephere


\subsection{Radiation hazard of energetic particle and the interaction with the planet for instance Moon and Mars}

Radiation hazard of the SEP 
- Space
- on the planetary environment
Radiation from GCR and the secondary particle generated by the GCR

- GCR interaction with regolith, lunar-
	- Generation of Neutron

The exploration of space has witnessed a surge in intensity, with an increasing number of countries aspiring to venture into this domain. Noteworthy examples include NASA's initiation of the Artemis mission, which aims to return to the Moon by 2024. Similarly, China has unveiled its plans to establish a lunar base on the lunar surface by the 2030s, while the European Space Agency (ESA) has also embarked on a lunar lander mission. Most recently, a Japanese lunar lander mission was launched; however, it regrettably encountered failure.

Under these circumstances, the study of solar energetic particles (SEPs) assumes greater significance. SEPs pose a significant radiation hazard for future human exploration on the lunar surface. The most hazardous SEP events have the potential to induce radiation increases of substantial magnitude.

\subsection{Motivition}
New instrument, new data, 
Now observation point.
new solar cycle, special solar Cycle
new solar minimum, special solar minimum
