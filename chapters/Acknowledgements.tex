%*******************************************************
% Acknowledgements
%*******************************************************
\refstepcounter{dummy}
\pdfbookmark[0]{Acknowledgements}{Acknowledgements}
\chapter*{Acknowledgements}

At this point, I would like to express my gratitude to everyone who have played an important role during the course of my PhD. Their assistence, guidance, support have been invaluable over the past five years. Without their help, I would not be where I am now.

First of all, I would like to thank my supervisor Prof. Robert F. Wimmer-Schweingruber for giving me the opportunity to join the group and work on the exciting projects - Chang'E-4/LND and SolO/EPD. He supported me to publish these results in scientific journals and present at multiple international conferences. I enjoyed discussing with him. Those discussions inspired me and made me becoming more productive.

I am also grateful to Prof. Jingnan Guo, who recommended me to Robert five years ago. She guided and helped me a lot on the LND projects. I hope I could have opportunities to continue our collaborations in future. Besides, I would like to thank Prof. Dresing Nina, who provided valuable instructions and commments on the SEP paper of LND. She kindly wrote me a strong reference letter for me for the postdoc application of Caltech. That recommendation must be one of the key factors that help me to be selected from multiple candidates.

Furthermore, many thanks are given to my lovely colleagues woring in those projects and in the Kiel group.  Firstly, I would like to thank my officemate Dr. Lars Berger, for his patience and kindness every time I asked him basic and stupid questions. These discussions with him was one of my best moments that I enjoyed most. Despite of that, I would like to show special thanks to him for his encouragement during the course of finishing my thesis in the last two months. By the way, I still can not enjoy your heavy metal music. 
Secondly, I would like to show my appreciation to Dr. Verena Heidrich-Meisner, Dr. Patrick K\"{u}hl, and Dr. Marquardt Johannes, for their proofreading of this thesis. They provided valuable suggestions and had largely improved the thesis.
Next, I would like to express my appreciation to my colleagues in the Kiel group and those tow projects for their helpful discussion. A list of people include the future Ph.D., Alexander Kollhoff, Chaoran Gu, Salman Khaksarighiri, Kr\"{o}hnke Henning, and Dr. Daniel Pacheco, Dr. Liu Yang, Dr. Johan von Forstner, Dr. Jia Yu, Henning Lohf,

I also want to thank my friends in Kiel, Jinru He, Lei Shao, Xuenan Li, Xiuming Sun, who have always supported me, taken care of me, and helped me survive the tough COVID times. I remember the time that we enjoyed the fried chicken wings and played Catan all night until the following day. Thanks those tough times, they made me stronger and more confident.

In the end I would like to thank my parents - Aifang Wang and Fazhen Xu - for their understanding and support during my studies. 
Specifically, I would like to express my deepest love and gratitude to my wife, Zhenlin Zhu. She always has my back abd I am so lucky to have her in my life.


Further acknowledgments to the online tools for thesis writing, including but not limited to Chatgpt, Grammarly, NotionAI, and DeepL. They are really powerful in correcting grammar mistakes.
 

% At this point, I would like to thank everyone who supported me during the course of my PhD studies and my work on MSL/RAD, Chang'E 4 LND as well as Solar Orbiter EPD.

% First of all, I am grateful to my supervisor Prof. Robert Wimmer-Schweingruber for the opportunity to work on these exciting projects, and his helpful advice. He also made it possible that these results could be published in scientific journals and presented at many international conferences. Sincere thanks also to Prof. Jingnan Guo, who supported my work since my bachelor's thesis, and who I wish all the best for her new position in China.

% Furthermore, thanks to all my colleagues in the three mission teams for their continued support and helpful discussions, including Zigong Xu, Alexander Kollhoff, and my officemate Christoph Terasa for the productive and enjoyable collaboration, e.g. on low- and high-level software for the Solar Orbiter mission, which will hopefully facilitate the EPD data analysis in the Kiel team for years to come, and to the rest of the Extraterrestrial Physics group at Kiel University. I also thank the group of Manuela Temmer and Astrid Veronig at the University of Graz and Mateja Dumbović at Hvar Observatory with whom I worked in close collaboration for many of the Forbush decrease studies, and who I enjoyed meeting regularly at the conferences in Vienna, Hvar and San Francisco.

% I additionally want to thank the bachelor and master students who supported my work during their Hiwi positions, Charlotte Büschel and Niklas Lundt, and three secondary school students, Joana Wanger, Lukas Abegg and Markus Arndt, who contributed to the data sets used in my studies during their internships in the ET group.

% I would like to thank Anne Fischer as well as Hanna Giese and Knud Schröter for their very thorough proofreading of this thesis and valuable suggestions, and my parents Kristina and Michael and my brother Julius for their moral support. 

% Last but not least, the data analysis presented in this thesis, the typesetting of the thesis itself and the generation of most of the figures were made possible by a number of open source software projects, including, but not limited to, the ones acknowledged below:

% \begin{refsection}[software.bib]
% 	\nocite{*}
% 	\newrefcontext[sorting=none]
% 	\printbibliography[heading=none]
% \end{refsection}