%*******************************************************
% Acknowledgements
%*******************************************************
\refstepcounter{dummy}
\pdfbookmark[0]{Acknowledgements}{Acknowledgements}
\chapter*{Acknowledgements}

At this point, I would like to express my gratitude to everyone who have played an important role during the course of my PhD. Their assistence, guidance, support have been invaluable over the past five years. Without their help, I would not be where I am now.

First of all, I would like to thanks my supervisor Prof. Robert F. Wimmer-Schweingruber for gaving me the opportunity to join his group and work on the exciting LND and SolO/EPD projects. He support me to publish these results in scientific journals and present at many international conferences. I like every discussion with him, which inspire me always and are productive.

I am also grateful to Prof. Jingnan Guo, who recommended me to Bob five years ago and worked with me on the LND projects during my first stage here. I hope I could work with her again in the future. I also would like to thank Prof. Dresing Nina, who worked instruct me on my first paper of LND and gave me the reference letter for my application of the postdoc position. That must be strong recommendation.

Furthermore, many thanks to my lovely colleagues in projects and in the group. My appreciation to my officemate Lars Berger, thanks for your patience and kindness every time I asked you those basic and stupid questions. Discussing with you was one of my best memory in Kiel. And thanks for your help and encouragement during the last two months with my thesis. By the way, I still can not enjoy your music. My appreciation to Verena Heidrich-Meisner, Patrick K\"{u}hl, and Marquardt, Johannes, for your proofreading of this thesis and valuable suggestions. I try my best to revise most of your comments, but unfortunately, not all of them. I would like to express my appreciation to my colleagues, Alexander Kollhoff, Daniel Pacheco, Liu Yang, Chaoran Gu, Salman Khaksarighiri, Kr\"{o}hnke Henning, Johan von Forstner, Jia Yu, Henning Lohf, for their helpful discussion.

I also want to thank my friends in Kiel, Jinru He, Lei Shao, Xuenan Li, Xiuming Sun, who have always supported me, taken care of me, and helped me survive the tough COVID times. I will remember the time when we enjoyed those fried chicken wings and played Catan all night until the following day. Thanks those tough times, they made me stronger and more confident.

In the end I would like to thank my parents Aifang Wang and Fazhen Xu for their understanding and support during my studies. 
Specifically, I would like to express my deepest love and gratitude to my wife, the future Dr. Zhu, Zhenlin Zhu. She has always got my back. I am so lucky to have her and love you for all my life.


Further acknowledgments to the online tools for thesis writing, including but not limited to Chatgpt, Grammarly, NotionAI, and DeepL. They are really powerful in correcting grammar mistakes.
 

% At this point, I would like to thank everyone who supported me during the course of my PhD studies and my work on MSL/RAD, Chang'E 4 LND as well as Solar Orbiter EPD.

% First of all, I am grateful to my supervisor Prof. Robert Wimmer-Schweingruber for the opportunity to work on these exciting projects, and his helpful advice. He also made it possible that these results could be published in scientific journals and presented at many international conferences. Sincere thanks also to Prof. Jingnan Guo, who supported my work since my bachelor's thesis, and who I wish all the best for her new position in China.

% Furthermore, thanks to all my colleagues in the three mission teams for their continued support and helpful discussions, including Zigong Xu, Alexander Kollhoff, and my officemate Christoph Terasa for the productive and enjoyable collaboration, e.g. on low- and high-level software for the Solar Orbiter mission, which will hopefully facilitate the EPD data analysis in the Kiel team for years to come, and to the rest of the Extraterrestrial Physics group at Kiel University. I also thank the group of Manuela Temmer and Astrid Veronig at the University of Graz and Mateja Dumbović at Hvar Observatory with whom I worked in close collaboration for many of the Forbush decrease studies, and who I enjoyed meeting regularly at the conferences in Vienna, Hvar and San Francisco.

% I additionally want to thank the bachelor and master students who supported my work during their Hiwi positions, Charlotte Büschel and Niklas Lundt, and three secondary school students, Joana Wanger, Lukas Abegg and Markus Arndt, who contributed to the data sets used in my studies during their internships in the ET group.

% I would like to thank Anne Fischer as well as Hanna Giese and Knud Schröter for their very thorough proofreading of this thesis and valuable suggestions, and my parents Kristina and Michael and my brother Julius for their moral support. 

% Last but not least, the data analysis presented in this thesis, the typesetting of the thesis itself and the generation of most of the figures were made possible by a number of open source software projects, including, but not limited to, the ones acknowledged below:

% \begin{refsection}[software.bib]
% 	\nocite{*}
% 	\newrefcontext[sorting=none]
% 	\printbibliography[heading=none]
% \end{refsection}