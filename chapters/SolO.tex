Together with the NASA \ac{PSP} launched in August 2018, the ESA \ac{SolO} mission ushers in a new era of solar and heliospheric physics. For the first time since the 1970s, these spacecraft will approach the Sun significantly closer than the orbit of Mercury --- \ac{PSP} has already set a new record with less than \SI{0.1}{\AU} solar distance at its most recent perihelion in September 2020. On the other hand, \ac{SolO}, which was launched in February 2020, will come very close to the Sun as well ($\sim\SI{0.28}{\AU}$), although it will stay far enough away to also allow for imaging observations through holes in its heatshield. In the extended mission phase, it is planned to incline the orbit of \ac{SolO} to also observe the poles of the Sun directly for the first time.

As part of the \ac{EPD} suite on \ac{SolO} \citep{RodriguezPacheco-2019-EPD}, the \acl{HET}\acused{HET} (\acs{HET}, \autoref{sec:solohet}) has been successfully commissioned and is providing some first measurements of high-energy charged particles. While a few \ac{SEP} events in the first 10 months of the mission did extend to the energies covered by \ac{HET} ($\gtrsim\SI{6}{\mega\electronvolt\per nuc}$ ions and $>\SI{450}{\kilo\electronvolt}$ electrons), \ac{HET} spent most of the time observing the \ac{GCR} background, as the solar activity was quite low during this time. As discussed in \autoref{sec:solohet}, \ac{HET} is also able to resolve short-term variations of \acp{GCR} with some of its data products. Consequently, some \acp{FD} could be measured, which were caused by \acp{CME} and \acp{CIR} that passed \ac{SolO} during its first orbit.

The first \ac{CME}-induced \ac{FD} seen at \ac{SolO} on April 19, 2020 is especially interesting, as it is a multispacecraft event that was also observed near Earth one day later during a close longitudinal alignment and with a radial separation of \SI{0.2}{\AU}. Measurements of the \ac{FD} near Earth have been taken by neutron monitors as well as the \ac{CRaTER} onboard the \ac{LRO}. The \ac{CME} was also seen at the BepiColombo spacecraft that was still close to Earth at this time, and may also have hit Venus, though no observations are available at this location due to the loss of contact with the Venus Express spacecraft since 2014. Furthermore, \acs{STEREO}-A was in an excellent position to provide a side view of the \ac{CME} with its remote sensing instruments. In the following publication, which was submitted to \textit{Astronomy \& Astrophysics} in November 2020, we describe the capabilities of \ac{HET}, present the \ac{FD} observed at \ac{SolO} and the corresponding observations near Earth, and investigate the radial evolution of the \ac{CME} by applying \acs{ForbMod} (see \autoref{sec:forbush}) to this event. Two other studies of the same event, have also been submitted to \textit{A\&A}: \textcite{Davies-2021} focus on the magnetic field observations, while \textcite{OKane-2021-SolO} investigate the solar source of the CME. All three papers will be published in the \textit{A\&A} ``Solar Orbiter First Results'' special issue in 2021.

In the process of this study, a new software implementation of the \ac{GCS} model \citep{Thernisien-2011-GCS} has been developed. Details about this tool can be found in \autoref{chp:GCS_Python}.

\newpage

The following article is reproduced from \textcite{Forstner-2020-SolO} with permission from Astronomy \& Astrophysics, \copyright The European Southern Observatory (ESO):\\

\pubcite{Forstner-2020-SolO}
\hfill Own contribution: 80\%

\newpage
\newcounter{includepdfpageAATwenty}

\todo[inline]{Insert Article}