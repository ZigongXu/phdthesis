Launched in February 2020, the ESA \ac{SolO} spacecraft, together with the NASA \ac{PSP} mission launched in August 2018, ushers in a new era in solar and heliospheric physics. For the first time since the 1970s, these spacecraft will approach the Sun significantly closer than the orbit of Mercury --- \ac{PSP} has already set a new record with less than \SI{0.1}{\AU} solar distance at its most recent perihelion in September 2020, while \ac{SolO} will stay far enough away from the Sun to also be able to make imaging observations. In the extended mission phase, it is planned to incline the orbit of \ac{SolO} to also observe the poles of the Sun for the first time.

As part of the \ac{EPD} suite on \ac{SolO} \citep{RodriguezPacheco-2019-EPD}, the \ac{HET} (\autoref{sec:solohet}) has been successfully commissioned and provided some first measurements of high-energy charged particles. While a few \ac{SEP} in the first 10 months of the mission did extend to the energies covered by \ac{HET} ($\gtrsim\SI{6}{\mega\electronvolt\per nuc}$ ions and $>\SI{450}{\kilo\electronvolt}$ electrons), the Sun was very quiet during most of this time so that \ac{HET} mainly observed the \ac{GCR} background. As discussed in Section \autoref{sec:solohet}, \ac{HET} is also able to measure short-term variations of the \ac{GCR} when using the appropriate data products, and consequently, some \acp{FD} have been measured, which were caused by \acp{CME} and \acp{CIR} that passed \ac{SolO} during its first orbit.

The first \ac{FD} seen at \ac{SolO} on April 19, 2020 is especially interesting, as it is a multispacecraft \ac{CME} that was also observed near Earth one day later during a close longitudinal alignment and with a radial separation of \SI{0.2}{\AU}. The \ac{CME} was also seen at the BepiColombo spacecraft that was still close to Earth at this time, and may also have hit Venus, though no observations at Venus are available due to the loss of contact with the Venus Express spacecraft in 2014. Furthermore, \ac{STEREO}-A was in a perfect position to provide a side view of the \ac{CME} with its remote sensing instruments. In the following publication, which was submitted to \textit{Astronomy \& Astrophysics} in November 2020, we describe the capabilities of \ac{HET}, present the \ac{FD} observed at \ac{SolO} and the corresponding observations near Earth, and investigate the radial evolution of the \ac{CME} by applying \acs{ForbMod} (see \autoref{sec:forbush}) to this event. Another study of this event, focusing on the magnetic field observations, has also been submitted by Emma E. Davies of Imperial College, and both will be published in the ``Solar Orbiter First Results'' special issue of \textit{A\&A} in 2021.

In the process of this study, a new implementation of the \ac{GCS} model \citep{Thernisien-2011-GCS} was developed. Details about this software can be found in \autoref{chp:GCS_Python}.

\newpage

The following article is reproduced from \textcite{Forstner-2020-SolO} with permission from Astronomy \& Astrophysics, \copyright The European Southern Observatory (ESO):\\

\pubcite{Forstner-2020-SolO}
\hfill Own contribution: 80\%

\newpage
\newcounter{includepdfpageAATwenty}

\todo[inline]{Insert Article}